\PassOptionsToPackage{numbers,sort&compress}{natbib}
\documentclass[MSc,french]{ulthese}
\usepackage{ragged2e} % \RaggedRight, \Centering, etc.
% ----------------------------------------------------------------------
% ENCODAGE, LANGUE & MICRO‑TYPOGRAPHIE

% ----------------------------------------------------------------------
\usepackage[T1]{fontenc} % codage de fonte moderne
\usepackage[utf8]{inputenc} % encodage de la source (pdfLaTeX uniquement)
\usepackage{csquotes} % guillemets « français »/‹ anglais › …
\usepackage{microtype} % protrusion & expansion
%

%----------------------------------------------------------------------
% POLICES
% ----------------------------------------------------------------------
\usepackage{lmodern} % Latin Modern — vectoriel + T1
% ----------------------------------------------------------------------
% MATHÉMATIQUES

% ----------------------------------------------------------------------
\usepackage{amsmath,amssymb,amsthm}
\usepackage{icomma} % espace fine après virgule decimale
% \usepackage{mathtools} % décommentez pour besoins avancés
% ----------------------------------------------------------------------
% TABLEAUX & NOMBRES
% ----------------------------------------------------------------------

\usepackage{booktabs,longtable,tabularx}
\usepackage{siunitx}
\sisetup{
	output-decimal-marker = {,},
	detect-all, % détecte famille/gras/italique
	per-mode = symbol % \si{\per\something}
}
% ----------------------------------------------------------------------
% GRAPHIQUES & FIGURES
% ----------------------------------------------------------------------
\usepackage{graphicx}
\usepackage{enumitem}
\usepackage{subcaption}
\usepackage{float} % placement précis [H]
\usepackage{placeins} % \FloatBarrier
% PGF/TikZ & PGFPLOTS
\usepackage{tikz}
\usetikzlibrary{arrows.meta,positioning,shapes.geometric}
\usepackage{pgfplots}
\pgfplotsset{compat=1.18}
\usepackage{tikz-cd}
% ----------------------------------------------------------------------
% BIBLIOGRAPHIE
% ----------------------------------------------------------------------
\usepackage{chapterbib} % bibliographies par chapitre (nécessite \bibliography)
% natbib est déjà chargé par la classe ulthese avec nos options ci‑dessus
% ----------------------------------------------------------------------
% HYPERLIENS & RÉFÉRENCES CROISÉES
% ----------------------------------------------------------------------
\usepackage{xcolor}
\usepackage[unicode,colorlinks=true,linkcolor=teal,citecolor=teal,urlcolor=teal,
bookmarksopen=true,bookmarksdepth=3]{hyperref}
\usepackage{cleveref} % après hyperref
% ----------------------------------------------------------------------
% UTILITAIRES
% ----------------------------------------------------------------------
\usepackage{etoolbox}

\usepackage{listings}
\lstset{
	language=R,
	basicstyle=\ttfamily\small,
	keywordstyle=\color{blue}\bfseries,
	stringstyle=\color{red},
	commentstyle=\color{green!60!black}\itshape,
	numbers=left,
	numberstyle=\tiny\color{gray},
	stepnumber=1,
	numbersep=5pt,
	backgroundcolor=\color{gray!10},
	showstringspaces=false,
	frame=single,
	tabsize=2,
	breaklines=false,
	breakatwhitespace=false,
	captionpos=b,
	escapeinside={\%*}{*)},
	morekeywords={library, here, fs, janitor, define_constants_gbd, generate_file_paths, generate_output_path, compute_logit_1e5, process_gbd_data, map_dfr, clean_names, filter, mutate, select, distinct, write_csv}
}



% ----------------------------------------------------------------------
% PATCHES
% ----------------------------------------------------------------------
\patchcmd{\chapter}{\par}{\par\vspace{1em}}{}{}
% ----------------------------------------------------------------------
% CONFIGURATION CSV (optionnel)
% ----------------------------------------------------------------------
\usepackage{csvsimple}
\csvset{
	separator = {comma},
	respect all,
	tabular = {r},
}
% ----------------------------------------------------------------------
% ENVIRONNEMENTS DE THÉORÈMES (noms français)
% ----------------------------------------------------------------------
\theoremstyle{definition}
\newtheorem{theorem}{Théorème}
\newtheorem{conj}{Conjecture}
\newtheorem{lemma}[theorem]{Lemme}
\newtheorem{corollary}[theorem]{Corollaire}
\newtheorem{proposition}[theorem]{Proposition}
\newtheorem{definition}{Définition}
\newtheorem{example}[theorem]{Exemple}
\newtheorem{remark}{Remarque}
\newtheorem{notation}[theorem]{Notation}
% ----------------------------------------------------------------------
% MACROS & NOTATIONS PERSONNALISÉES


% ----------------------------------------------------------------------


% Extensions graphiques (pdfLaTeX)
\DeclareGraphicsExtensions{.pdf,.png,.jpg,.eps}
% ----------------------------------------------------------------------
% MÉTADONNÉES PDF & PAGE TITRE
% ----------------------------------------------------------------------
\hypersetup{
	pdftitle = {Application de la géométrie de l’information à l’étude épidémiologique des maladies respiratoires chroniques},
	pdfauthor = {José Manuel Rodríguez Caballero}
}
% Info de page‑titre (macros de la classe)
\titre{Application de la géométrie de l’information à l’étude épidémiologique des maladies respiratoires chroniques}
\soustitre{L'effet de la toxicité atmosphérique}
\auteur{José Manuel Rodríguez Caballero}
\programme{Maîtrise en statistique}
\direction{M'Hamed Lajmi Lakhal Chaieb}
\codirection{Karim Barigou}
\annee{2025}
% ----------------------------------------------------------------------
% DOCUMENT
% ----------------------------------------------------------------------
\begin{document}
	%\frontmatter % pages liminaires
	%\frontispice % page de couverture officielle
	%\section{Résumé}

Ce mémoire examine l'impact de la concentration régionale d'arsenic dans les particules fines PM$_{2{,}5}$ sur l'incidence et la prévalence des maladies respiratoires chroniques (MRC), en tenant compte des variations selon l'âge et le sexe. Motivé par les liens établis entre l'ingestion d'arsenic et les affections respiratoires, ainsi que par les effets synergiques avec les PM$_{2{,}5}$, l'étude intègre une analyse préliminaire des interactions âge-sexe sans considération de l'arsenic, interprétées à travers la théorie évolutive de Geodakyan, qui souligne une plus grande variabilité chez les mâles pour favoriser l'adaptation environnementale. Les implications s'étendent à l'actuariat, pour affiner les modèles de primes d'assurance santé, et à la santé publique, pour une allocation ciblée des ressources.

Sur le plan empirique, les analyses révèlent que la somme cumulée des distances de Fisher-Rao entre trajectoires consécutives des paramètres de distribution des incidences est systématiquement plus élevée chez les hommes que chez les femmes, avec des résultats similaires par la divergence de Kullback-Leibler ; ces motifs sont consistants entre pays. La relation entre exposition à l'arsenic et incidence dépend de l'âge, avec une inversion chez les sujets très âgés due à un biais de diagnostic, tandis que la prévalence augmente linéairement après transformations. Méthodologiquement, l'approche repose sur la géométrie de l'information (distance de Fisher-Rao, divergence de Kullback-Leibler), la taille d'effet par le \( g \) de Cohen pour les proportions, et le bootstrapping pour estimer la variabilité.

Interprétés dans le cadre de la théorie de Geodakyan, ces résultats contribuent au débat sur ses fondements biologiques et ouvrent des perspectives pratiques en actuariat (ajustement des primes), épidémiologie (extension à d'autres pathologies), environnement (liens avec le changement climatique augmentant les concentrations d'arsenic aérien par la sécheresse) et sciences politiques (distribution optimisée des ressources sanitaires). Le mémoire est structuré en chapitres couvrant les préliminaires mathématiques, les données, la méthodologie, les résultats, la discussion, les conclusions et des annexes méthodologiques.
	%\section{Abstract}

This master's dissertation investigates sex-related differences in the trajectories of statistical parameters and the relationship between atmospheric exposure (PM2.5 — proxy for arsenic) and chronic respiratory diseases, combining information geometry methods, effect size indices, and epidemiological analyses on multi-country datasets.

\paragraph{Methods.} To quantify temporal variations of parameters $(\mu,\sigma)$, I use the Fisher–Rao distance applied to parametric families, complemented by the Kullback–Leibler divergence as an alternative measure. Effect sizes (Cohen’s g) are computed by age group; exposure–prevalence relationships are modeled via linear regressions on log- and logit-transformed prevalence. Diagnostics include Shapiro–Wilk tests for residual normality and corrections for multiple comparisons.

\paragraph{Main results.} The cumulative sum of successive Fisher–Rao distances is consistently higher for men than for women, a property confirmed across several countries; analogous findings emerge with the KL divergence. Analysis of effect sizes (Cohen’s g) shows male predominance during childhood, female predominance during adolescence and young adulthood, followed by inter-country variability after age 50, with a general trend toward male predominance. For arsenic exposure (PM2.5), incidence varies with age: around 47 years (±2), incidence increases with arsenic concentration among men, while in very old men (\~92 ±2 years) the relationship reverses — a bias explained by accumulation and earlier diagnosis. In contrast, prevalence increases robustly with pollution; the prevalence–pollution relationship appears linear after log / logit transformations. Residual normality diagnostics are highly satisfactory, subject to multiple-testing corrections.

\paragraph{Interpretation and implications.} I interpret these empirical patterns in light of Geodakyan’s biological theory, which offers a framework to explain observed sex differences. The results have practical implications for actuarial science (health insurance pricing by age/sex/pollution), epidemiology (transferable methods to other diseases), and environmental health policy (targeted resource allocation). Data and R scripts are provided in the appendix. Future work could extend the study to other pollutants and implement causal longitudinal approaches.

	%\tableofcontents
	%\cleardoublepage
	%\listoftables
	%\cleardoublepage
	%\listoffigures
	%\cleardoublepage
	%\section*{Remerciements}
Je remercie chaleureusement mon directeur, M’Hamed Lajmi Lakhal Chaieb, et mon codirecteur, Karim Barigou, pour leur encadrement, leurs conseils et leur disponibilité tout au long de ce mémoire.

Je tiens également à exprimer ma gratitude à Monsieur Charles Brindamour, donateur de la Bourse d’excellence Charles Brindamour, dont le soutien financier et la reconnaissance académique ont grandement contribué à la réalisation de ce travail.

Je suis reconnaissant envers mes collègues de maîtrise pour leurs échanges enrichissants, ainsi qu’à ma famille et mes proches pour leur soutien constant et leurs encouragements.
	%\section*{Avant-propos}
Après avoir déménagé à la ville de Québec pour poursuivre mes études à l’Université Laval, j’ai observé que, particulièrement durant l’été, la qualité de l’air était vraiment mauvaise et dangereuse pour la santé. Cette constatation m’a amené à me poser la question suivante : l’effet de la pollution atmosphérique sur la santé est-il négligeable ou non ? Ce mémoire offre une réponse partielle à cette interrogation générale. J’espère que ma contribution favorisera une prise de conscience écologique accrue et stimulera une volonté politique visant à réduire la pollution atmosphérique. Dans les situations où celle-ci s’avère inévitable, telles que les incendies naturels, il convient d’adopter des mesures pour limiter l’exposition humaine, par exemple en recourant à des masques et à des purificateurs d’air. De plus, compte tenu de l’effet du changement climatique sur la concentration atmosphérique d’arsenic à cause de la sécheresse, j’espère que cette thèse motivera la recherche sur les effets directs et indirects du changement climatique sur la santé humaine.
	%\mainmatter % corps du document
	%\chapter{Introduction}
\label{chap:intro}

\section{Contexte et motivation}

Les maladies respiratoires chroniques (MRC), incluant la maladie pulmonaire obstructive chronique (MPOC), la pneumoconiose, l'asthme, les maladies pulmonaires interstitielles et la sarcoïdose pulmonaire, constituent des causes majeures de morbidité mondiale, avec une augmentation de 39,5 \% du nombre total de cas entre 1990 et 2017, bien que les taux d'incidence et de prévalence standardisés selon l'âge aient diminué \citep{Xie2020}. Ces pathologies présentent des variations significatives selon l'âge, le sexe et les régions.

Certaines maladies respiratoires, telles que le cancer du poumon, la toux chronique et la bronchite, ont été associées à l’ingestion d’arsenic, notamment par l’eau potable ou la fumée de tabac \citep{Parvez2009ArsenicCOPD, Smith2006, RAMSEY2023381, Sengupta14082025}. Selon \citep{RAMSEY2023381}, parmi les toxiques environnementaux, l’arsenic se distingue par sa capacité à provoquer des affections respiratoires malignes et non malignes principalement par voie orale, plutôt que par inhalation. Cette particularité en fait un agent pathogène atypique pour l’appareil respiratoire.

Il serait pertinent d’examiner si, malgré son mode d’action principalement oral, la concentration d’arsenic présente dans les particules fines PM$_{2{,}5}$ en suspension dans l’air constitue un facteur prédictif de maladies respiratoires chroniques. Il est bien établi que les particules fines PM$_{2{,}5}$ pénètrent profondément dans les poumons, irritent et corrodent la paroi alvéolaire, altérant ainsi la fonction pulmonaire \citep{Xing2016PM25}. De plus, l’exposition simultanée aux PM$_{2{,}5}$ (par voie intratrachéale) et à l’arsenic (par voie orale) induit une inflammation pulmonaire chez les souris de laboratoire nettement plus élevée que celle provoquée par chacun de ces agents pris séparément, suggérant un effet synergique \citep{RivasSantiago2024}.

Le présent mémoire s'intéresse principalement à l'effet de la concentration régionale d'arsenic dans les PM$_{2{,}5}$ sur l'incidence et la prévalence des maladies respiratoires chroniques (MRC), qui varient selon les tranches d'âge et le sexe. Une étude préliminaire de l'interaction entre l'âge et le sexe est également réalisée, sans considérer l'impact de l'arsenic dans les PM$_{2{,}5}$. Les différences sexuelles observées sont interprétées dans le cadre de la théorie de Geodakyan, qui postule une plus grande variabilité chez les mâles pour favoriser l'adaptation \citep{Geodakyan18082015} :

\begin{quote}
	\emph{Tout système s'adaptant à un environnement variable se divise en deux sous-systèmes conjugués, spécialisés selon les tendances conservatrices et opératoires de l'évolution, ce qui augmente la stabilité du système dans son ensemble.}
\end{quote}

Ces questions scientifiques ont des implications directes pour l'actuariat, où l'adaptation des primes d'assurance santé selon l'âge, le sexe et le niveau de pollution régionale pourrait améliorer la précision des modèles \citep{bolviken2014computation}. En santé publique, elles impliquent une allocation ciblée des ressources, plutôt qu'une répartition uniforme, afin d'optimiser les interventions préventives et thérapeutiques \citep{read2010gender}.

\section{Contributions}

Ce mémoire apporte plusieurs contributions dans l'application de la géométrie de l'information à l'étude épidémiologique des maladies respiratoires chroniques.

Sur le plan empirique, les analyses montrent que la somme cumulée des distances de Fisher-Rao entre trajectoires consécutives des paramètres de distribution (moyenne et variance) des incidences est systématiquement plus élevée chez les hommes que chez les femmes, à quelques exceptions près. Cette propriété, observée dans plusieurs pays, suggère une base biologique. Un résultat analogue est obtenu en utilisant la divergence de Kullback-Leibler. Par ailleurs, les variations discrètes (« augmente », « diminue ») par tranche d'âge sont remarquablement consistantes entre pays. La relation entre l'exposition à l'arsenic et l'incidence apparaît dépendante de l'âge, avec une inversion chez les sujets très âgés attribuable à un biais de diagnostic, tandis que la prévalence augmente linéairement avec l'exposition après transformations appropriées.

Sur le plan méthodologique, l'approche mobilise des outils de géométrie de l'information, tels que la distance de Fisher-Rao et la divergence de Kullback-Leibler, pour quantifier les différences sexuelles et environnementales dans les distributions d'incidence. La taille d'effet est évaluée à l'aide du \( g \) de Cohen pour les proportions, et des méthodes de rééchantillonnage comme le bootstrapping sont utilisées pour estimer la variabilité.

Ces observations sont interprétées à la lumière de la théorie évolutive de Geodakyan, qui postule que la plus grande variabilité masculine constitue un mécanisme d'adaptation environnementale, contribuant ainsi au débat sur la validité de cette théorie.

Enfin, les implications pratiques de ces résultats touchent plusieurs domaines : l'actuariat (définition de critères pour l'ajustement des primes santé), l'épidémiologie (extension des méthodes à d'autres pathologies), l'environnement (lien entre changement climatique et augmentation des concentrations d'arsenic dans l'air due à la sécheresse), et les sciences politiques (distribution ciblée des ressources en santé publique).

\section{Plan du mémoire}

Le mémoire est structuré comme suit.

Le \textbf{chapitre~\ref{chap:math}} présente les préliminaires mathématiques, incluant la divergence de Kullback-Leibler, la distance de Fisher-Rao, la taille d'effet et le bootstrapping.

Le \textbf{chapitre~\ref{chap:data}} est consacré à la présentation des données utilisées dans l’étude.

Le \textbf{chapitre~\ref{chap:methods}} détaille la méthodologie de la recherche.

Le \textbf{chapitre~\ref{chap:results}} expose les résultats empiriques issus des analyses de données.

Le \textbf{chapitre~\ref{chap:discussion}}, dédié à la discussion, interprète ces résultats en lien avec la théorie de Geodakyan et explore les implications pratiques.

Enfin, le \textbf{chapitre~\ref{chap:conclusion}}, consacré aux conclusions, synthétise les contributions du mémoire et propose des perspectives pour des travaux futurs.

Des \textbf{annexes} fournissent des figures supplémentaires ainsi que des détails méthodologiques.
	%\chapter{Préliminaires mathématiques}
\label{chap:math}
Ce chapitre présente brièvement les outils mathématiques utilisés dans les développements ultérieurs : la divergence de Kullback–Leibler, la géométrie de l'information (métrique de Fisher et distance de Fisher–Rao), ainsi que quelques repères sur la notion de taille d'effet.

\section{Divergence de Kullback–Leibler}
Pour deux lois de probabilité $P$ et $Q$ ayant respectivement pour densités $p$ et $q$, la divergence de Kullback–Leibler est définie par \citep{ay2017information} :
\begin{equation}
	D_{KL}(P\|Q) = \int p(x)\log\frac{p(x)}{q(x)}\,dx.
\end{equation}
Quelques propriétés fondamentales :
\begin{enumerate}
	\item $D_{KL}(P\|Q) \ge 0$ ;
	\item $D_{KL}(P\|Q) = 0$ si et seulement si $P = Q$ presque partout ;
	\item La divergence est asymétrique : $D_{KL}(P\|Q) = D_{KL}(Q\|P)$ n'est pas toujours vraie.
\end{enumerate}
Dans le cas de lois normales univariées $P = \mathcal{N}(\mu_1, \sigma_1^2)$ et $Q = \mathcal{N}(\mu_2, \sigma_2^2)$, on a \citep{belov2011distributions} :
\begin{equation}
	D_{\mathrm{KL}}(\mu_{1},\sigma_{1};\mu_{2},\sigma_{2})
	= \ln\left(\frac{\sigma_{2}}{\sigma_{1}}\right)
	+ \frac{\sigma_{1}^{2}+(\mu_{2}-\mu_{1})^{2}}{2\sigma_{2}^{2}} - \tfrac{1}{2}.
	\label{eq:kl_normal}
\end{equation}

\section{Géométrie de l'information}
La géométrie de l'information traite les familles de distributions de probabilité comme des variétés riemanniennes, en utilisant des outils de la géométrie différentielle pour étudier leurs propriétés \citep{ay2017information}.

\subsection{Métrique de Fisher}
Pour une famille paramétrique de densités $p(x;\theta)$, la matrice d'information de Fisher, qui définit la métrique riemannienne, est donnée par :
\begin{equation}
	g_{ij}(\theta) = \mathbb{E}_{\theta}\!\left[\partial_{i}\log p(X;\theta)\,\partial_{j}\log p(X;\theta)\right].
\end{equation}

\subsection{Distance de Fisher–Rao}
La distance de Fisher–Rao $d_{\mathrm{FR}}(\theta_1,\theta_2)$ est la longueur minimale d’une courbe reliant $\theta_1$ à $\theta_2$ selon la métrique de Fisher.

Pour deux lois normales univariées $\mathcal{N}(\mu_1, \sigma_1^2)$ et $\mathcal{N}(\mu_2, \sigma_2^2)$, la distance de Fisher–Rao est \citep{nielsen2023simple} :
\begin{equation}
	d_{\mathrm{FR}}\big((\mu_1,\sigma_1),(\mu_2,\sigma_2)\big)
	= \sqrt{2}\,\operatorname{acosh}\!\left(1 + \frac{(\mu_2 - \mu_1)^2 + 2(\sigma_2 - \sigma_1)^2}{4\sigma_1\sigma_2}\right).
	\label{eq:fr_normal}
\end{equation}

\section{Taille de l'effet}
La taille de l'effet quantifie l'ampleur d'une différence indépendamment de la taille de l’échantillon. Elle est utile pour interpréter la pertinence pratique d’un résultat statistique.

\subsection{Proportions (g de Cohen)}
Pour une proportion $P$, la taille d’effet selon Cohen est définie par :
\begin{equation}
	g = P - \tfrac{1}{2}.
\end{equation}
Selon les seuils proposés par Jacob Cohen \citep{cohen1992power}, on considère :
\begin{enumerate}
	\item $|g| \approx 0{,}05$ : effet faible ;
	\item $|g| \approx 0{,}15$ : effet moyen ;
	\item $|g| \approx 0{,}25$ : effet fort.
\end{enumerate}

\section{Bootstrapping}
Le bootstrapping est une méthode de rééchantillonnage statistique qui permet d'estimer la variabilité d'un estimateur ou de construire des intervalles de confiance à partir d'un seul échantillon, sans supposer une forme paramétrique spécifique pour la distribution sous-jacente \citep{tibshirani1993introduction}.

L'approche non paramétrique consiste à générer de multiples échantillons bootstrap en tirant avec remplacement $n$ observations à partir de l'échantillon original de taille $n$. Pour une statistique d'intérêt $\hat{\theta}$, on calcule $\hat{\theta}^*$ pour chaque échantillon bootstrap. La distribution empirique des $\hat{\theta}^*$ approxime la distribution d'échantillonnage de $\hat{\theta}$.

Par exemple, l'écart-type bootstrap d'un estimateur est :
\begin{equation}
	\widehat{\mathrm{SE}}_{\mathrm{boot}} = \sqrt{\frac{1}{B-1} \sum_{b=1}^B (\hat{\theta}^{*b} - \bar{\hat{\theta}}^*)^2},
\end{equation}
où $B$ est le nombre d'échantillons bootstrap, $\hat{\theta}^{*b}$ est la statistique pour le $b$-ième bootstrap, et $\bar{\hat{\theta}}^*$ est la moyenne des $\hat{\theta}^{*b}$.

Des variantes incluent le bootstrap paramétrique, où les rééchantillons sont tirés d'une distribution ajustée à l'échantillon original.
	%\chapter{Présentation et nettoyage des données}
\label{chap:data}

\section{Données de l'EPA}

\subsection{Données brutes et nettoyage}

Les données brutes, issues de l'Environmental Protection Agency (EPA) \citep{epa_air_quality_system_2024} des États-Unis, consistent en des mesures quotidiennes de concentrations d'arsenic dans les particules fines PM$_{2,5}$, collectées de 1990 à 2019. Chaque fichier annuel, tel que \texttt{daily\_HAPS\_2017.csv}, inclut des enregistrements avec des identifiants géographiques (codes d'État, de comté et de site), des coordonnées spatiales, le paramètre mesuré (« Arsenic PM2.5 LC »), la durée d'échantillonnage, la date locale, l'unité de mesure (microgrammes par mètre cube, notée LC pour « local conditions ») et la valeur maximale observée sur 24 heures. Ces données exhibent une variabilité due aux conditions météorologiques, aux sources d'émission locales et aux protocoles de mesure.

Le nettoyage exclut les régions comme Puerto Rico, les Îles Vierges et le Mexique, pour se concentrer sur les 50 États des États-Unis et le District of Columbia, compatibles avec les données de santé publique. Les observations sont filtrées pour retenir uniquement le paramètre « Arsenic PM2.5 LC » en microgrammes par mètre cube sur 24 heures. Le nom « District Of Columbia » est corrigé en « District of Columbia » pour uniformité.

Les données annuelles sont concaténées en un ensemble unique, avec renommage des variables : région (nom de l'État ou DC) et valeur (concentration maximale quotidienne).

Pour estimer la moyenne des concentrations par région, une approche bootstrap non paramétrique est appliquée. L'ensemble des données (observations quotidiennes de 1990 à 2019) est rééchantillonné avec remplacement 1\,000 fois ($B = 1\,000$), en utilisant des graines pseudo-aléatoires pour la reproductibilité. Pour chaque rééchantillon, les valeurs sont groupées par région et le logarithme (en base 10) de la moyenne arithmétique est calculé. Pour chaque région, l'estimateur du paramètre $\mu$ est la moyenne des 1\,000 logarithmes des moyennes bootstrap, offrant un estimateur ponctuel robuste de la concentration typique, atténuant la variabilité temporelle et les valeurs aberrantes. L'écart-type des logarithmes des moyennes bootstrap fournit une estimation du paramètre $\sigma$. Le test de normalité de Shapiro-Wilk est appliqué à la distribution des logarithmes des moyennes bootstrap pour évaluer la normalité de la distribution d'échantillonnage.

Ces résultats par région ($\mu$, $\sigma$, le nombre d'échantillons bootstrap $B$, statistique $W$ et valeur $p$ de Shapiro-Wilk) sont enregistrés dans le fichier :
\begin{center}
	\texttt{arsenic\_pm2.5\_locations.csv}
\end{center}

\subsection{Données nettoyées}

Soit $\mathcal{S}$ l'ensemble des régions d'un pays donné. Pour les États-Unis, chaque région est un État ou DC. Pour chaque région $s \in \mathcal{S}$, on considère la variable aléatoire $C_s \sim \mathcal{N}(\mu_s, \sigma_s^2)$, représentant le logarithme (en base 10) de la moyenne arithmétique de la concentration quotidienne d'arsenic PM$_{2,5}$. Les paramètres $\mu_s$ sont visualisés sur la carte de la Figure \ref{fig:arsenic-map}, qui illustre les concentrations moyennes par État. Les paramètres $(\mu_s, \sigma_s)$ sont représentés dans la Figure \ref{fig:arsenic_Poincare}.

\begin{figure}[H]
	\centering
	\includegraphics[width=1\textwidth]{figures/arsenic_map/arsenic_map.png}
	\caption{Concentrations moyennes d'arsenic PM$_{2.5}$ par État aux États-Unis (1990-2019).}
	\label{fig:arsenic-map}
\end{figure}

\begin{figure}[H]
	\centering
	\includegraphics[width=1\textwidth]{figures/arsenic_Poincare/arsenic_Poincare.png}
	\caption{Paramètres $(\mu, \sigma)$ par État aux États-Unis (1990-2019).}
	\label{fig:arsenic_Poincare}
\end{figure}

\section{Données du GBD-2021}
\label{sec:gbd}

L'incidence est le nombre de nouveaux cas pour 100\,000 éléments de la population satisfaisant des caractéristiques données. La prévalence est le nombre de cas existants pour 100\,000 éléments de la population satisfaisant des caractéristiques données \citep{rothman2024epidemiology, kleinbaum1991epidemiologic}.

\subsection{Données brutes et nettoyage}

Les données brutes proviennent de l'étude Global Burden of Disease (GBD) 2021 de l'Institute for Health Metrics and Evaluation (IHME) \citep{IHME2024GBD}, et consistent en des estimations d'incidence par 100\,000 habitants, avec bornes inférieures et supérieures d'intervalles de confiance à 95\,\%, pour les États-Unis. Ces estimations sont stratifiées par région (États ou district), année (de 1990 à 2021), tranche d'âge (chacune contenant 5 ans) et sexe (mâle et femelle). Les données sont réparties sur plusieurs fichiers CSV, chacun couvrant une partie des estimations.

Le nettoyage commence par la lecture et la concaténation des fichiers pour le pays « US ». Les tranches d'âge catégorielles sont associées à leurs points moyens (par exemple, « <5 years » à 2, « 5-9 years » à 7, etc.) pour obtenir des valeurs entières. Les années sont converties en entiers.

Pour modéliser l'incidence, une transformation log-odds en base 10 est appliquée aux bornes inférieures et supérieures : $\log_{10} \left( \frac{x}{10^5 - x} \right)$, où $x$ est la valeur par 100\,000. Sur cette échelle transformée, la moyenne $\mu$ est estimée comme la moyenne des bornes transformées, et l'écart-type $\sigma$ comme $(U - L) / (2 \times z_{0,975})$, où $U$ et $L$ sont les bornes transformées, en supposant un intervalle de confiance à 95\,\% pour une distribution normale et $z_{0,975} \approx 1{,}96$ est le quantile $0,975$ de la distribution normale standard.

Les colonnes finales incluent le pays, la région, l'année, l'âge (point moyen d'une tranche de 5 ans), le sexe, $\mu$ et $\sigma$. L'ensemble nettoyé est enregistré dans le fichier :
\begin{center}
	\texttt{IHME-GBD\_2021\_CLEAN\_incidence.csv}
\end{center}

\subsection{Données nettoyées}

Soient $\mathcal{T}$, $\mathcal{A}$ et $\mathcal{G}$ les ensembles des années, tranches d'âges et sexes, respectivement. Pour chaque combinaison de région $s \in \mathcal{S}$, année $t \in \mathcal{T}$, âge $a \in \mathcal{A}$ et sexe $g \in \mathcal{G}$ aux États-Unis, la variable aléatoire transformée $I_{s,t,a,g}$ (log-cotes de l'incidence) suit une distribution $\mathcal{N}(\mu_{s,t,a,g}, \sigma_{s,t,a,g}^2)$.
	%\chapter{Méthodes}
\label{chap:methods}

\section{L'effet de l'âge stratifié par le sexe}

Les approches méthodologiques reposent sur une modélisation géométrique raffinée, appliquée aux données d'incidence des maladies respiratoires chroniques issues de GBD-2021, stratifiées par âge et sexe. Elles s'inscrivent dans le cadre probabiliste défini au chapitre sur les données, où chaque quadruplet $(s,t,a,g) \in \mathcal{S} \times \mathcal{T} \times \mathcal{A} \times \mathcal{G}$ est associé à une variable aléatoire $I_{s,t,a,g} \sim \mathcal{N}(\mu_{s,t,a,g}, \sigma_{s,t,a,g}^2)$ représentant la transformée logit (en base 10) du taux d'incidence.

\subsection{Agrégation de données}

Soit $n = \# (\mathcal{S} \times \mathcal{T})$ le nombre d'événements spatio-temporels. Pour chaque paire $(a,g) \in \mathcal{A} \times \mathcal{G}$, on considère $B = 100\,000$ échantillons avec remplacement de $(s,t) \in \mathcal{S} \times \mathcal{T}$, où le $b$-ème échantillon est noté
\begin{equation}
(s_1^{(b)}, t_1^{(b)}), (s_2^{(b)}, t_2^{(b)}), \dots, (s_n^{(b)}, t_n^{(b)}),
\end{equation}
pour $b = 1, 2, \dots, B$. Pour le $b$-ème échantillon, on calcule la moyenne :
\begin{equation}
I_{a,g}^{(b)} = \frac{1}{n} \sum_{i=1}^n I_{s_i^{(b)}, t_i^{(b)}, a, g}^{(b,i)},
\end{equation}
où $I_{s_i^{(b)}, t_i^{(b)}, a, g}^{(b,i)}$ est une réalisation de $I_{s_i^{(b)}, t_i^{(b)}, a, g}$. On calcule ensuite la moyenne $\mu_{a,g}$ et l'écart-type $\sigma_{a,g}$ de la suite $I_{a,g}^{(1)}, I_{a,g}^{(2)}, \dots, I_{a,g}^{(B)}$. Le taux d'incidence $i_{a,g}$ est obtenu par l'inverse de la transformation logit (en base 10) :
\begin{equation}
i_{a,g} = \frac{10^5}{1 + 10^{-\mu_{a,g}}},
\end{equation}
avec des intervalles de confiance à 95\,\% dérivés des quantiles empiriques (2,5\,\% et 97,5\,\%) des valeurs simulées transformées, assurant une quantification précise de l'incertitude sur l'échelle originale des taux par 100\,000 habitants.

\subsection{Dynamique hyperbolique}

Les couples $(\mu_{a,g}, \sigma_{a,g})$ sont projetés dans le demi-plan hyperbolique de Poincaré $\mathbb{H} = \{ z \in \mathbb{C} \mid \Im(z) > 0 \}$, par $z_{a,g} = \mu_{a,g} + i \sigma_{a,g}$. Cette représentation exploite la structure riemannienne de l'espace des distributions normales univariées, modélisé comme une variété hyperbolique. Pour chaque sexe $g \in \mathcal{G}$, la fonction $a \mapsto (\mu_{a,g}, \sigma_{a,g})$ est interprétée comme la trajectoire d'une particule sur cette variété, avec la tranche d'âge $a$ jouant le rôle du temps.

\subsection{Dynamique symbolique}

Pour explorer les variations qualitatives, des indicateurs directionnels sont introduits sur la base des différences entre tranches d'âge consécutives (espacées de 5 ans) :
\begin{eqnarray}
	S_{a,g}^{(\mu)} &=& \mathrm{sign}(\mu_{a,g} - \mu_{a-5,g}), \\
	S_{a,g}^{(\sigma)} &=& \mathrm{sign}(\sigma_{a,g} - \sigma_{a-5,g}),
\end{eqnarray}
pour $a \geq 7$. Ces signes binaires, calculés pour chaque pays, mettent en évidence des motifs qualitatifs dans les données, moins apparents dans les analyses continues.

\subsection{Distance de Fisher-Rao}

La dissimilarité entre deux lois $\mathcal{N}(\mu_1, \sigma_1^2)$ et $\mathcal{N}(\mu_2, \sigma_2^2)$ est mesurée par la distance de Fisher-Rao, donnée par la formule \eqref{eq:fr_normal} du chapitre \ref{chap:math}, qui capture les géodésiques de la variété statistique. Les longueurs totales des trajectoires le long des âges, agrégées par sexe, quantifient les évolutions globales : une distance plus grande indique une évolution plus marquée, \emph{ceteris paribus}.

\subsection{Divergence de Kullback-Leibler}

En complément, la dissimilarité est évaluée par la divergence de Kullback-Leibler, dirigée de la première vers la seconde, définie par \eqref{eq:kl_normal} du chapitre \ref{chap:math}. Cette mesure asymétrique quantifie l'information perdue lors de l'approximation d'une distribution par une autre. Les divergences totales le long des trajectoires par âges, agrégées par sexe, évaluent les évolutions directionnelles : une valeur plus élevée signale une asymétrie accrue dans les changements, \emph{ceteris paribus}.

\subsection{Comparaisons transnationales}

Les résultats pour les États-Unis sont comparés à ceux du Royaume-Uni, de la Norvège et de l'Italie, en évaluant les divergences dans les trajectoires hyperboliques, les motifs symboliques, la distance de Fisher-Rao et la divergence de Kullback-Leibler. Cette analyse, ancrée dans les mêmes hypothèses probabilistes, identifie des invariants ou hétérogénéités liées à des contextes socio-économiques distincts.

\section{L'effet du sexe stratifié par l'âge}

Les approches méthodologiques reposent sur une modélisation géométrique raffinée, appliquée aux données d'incidence des MRC issues de GBD-2021, stratifiées par âge. Elles s'inscrivent dans le cadre probabiliste défini au chapitre sur les données, où chaque quadruplet $(s,t,a) \in \mathcal{S} \times \mathcal{T} \times \mathcal{A}$ est associé à des variables aléatoires pour les mâles et les femelles.

\subsection{Agrégation de données}

La statistique $g$ de Cohen est définie par la formule \eqref{eq:g_cohen} du chapitre \ref{chap:math}. Soit $n = \# (\mathcal{S} \times \mathcal{T})$. Pour chaque âge $a \in \mathcal{A}$, on considère $B = 100\,000$ échantillons avec remplacement de $(s,t) \in \mathcal{S} \times \mathcal{T}$, où le $b$-ème échantillon est noté
\begin{equation}
(s_1^{(b)}, t_1^{(b)}), (s_2^{(b)}, t_2^{(b)}), \dots, (s_n^{(b)}, t_n^{(b)}).
\end{equation}
Pour le $b$-ème échantillon, on génère les réalisations $I_{s_i^{(b)}, t_i^{(b)}, a, \mathrm{mâle}}^{(b,i)}$ et $I_{s_i^{(b)}, t_i^{(b)}, a, \mathrm{femelle}}^{(b,i)}$ pour $i = 1, \dots, n$. On calcule ensuite la statistique g de Cohen $g_a^{(b)}$ à partir de la formule 

\begin{equation}
	g_a^{(b)} = \frac{1}{2n}\sum_{i=1}^n \mathrm{sign} \left(I_{s_i^{(b)}, t_i^{(b)}, a, \mathrm{mâle}}^{(b,i)} -
	I_{s_i^{(b)}, t_i^{(b)}, a, \mathrm{femelle}}^{(b,i)}
	 \right).
\end{equation}

La valeur finale est
\begin{equation}
g_a = \frac{1}{B} \sum_{b=1}^B g_a^{(b)}.
\end{equation}
La distribution bootstrap de $g_a$ est approximée par $\mathcal{N}(\mu_a, \sigma_a^2)$, avec $\mu_a$ et $\sigma_a$ estimés à partir des $g_a^{(b)}$. Les intervalles de confiance à 95\,\% sont dérivés des quantiles empiriques (2,5\,\% et 97,5\,\%) des $g_a^{(b)}$.

\subsection{Dynamique hyperbolique}

Les couples $(\mu_a, \sigma_a)$ sont projetés dans le demi-plan hyperbolique de Poincaré $\mathbb{H} = \{ z \in \mathbb{C} \mid \Im(z) > 0 \}$, par $z_a = \mu_a + i \sigma_a$. La fonction $a \mapsto (\mu_a, \sigma_a)$ est interprétée comme la trajectoire d'une particule sur cette variété hyperbolique.

\subsection{Dynamique symbolique}

Des indicateurs directionnels sont introduits sur la base des différences entre tranches d'âge consécutives :
\begin{eqnarray}
	S_a^{(\mu)} &=& \mathrm{sign}(\mu_a - \mu_{a-5}), \\
	S_a^{(\sigma)} &=& \mathrm{sign}(\sigma_a - \sigma_{a-5}),
\end{eqnarray}
pour $a \geq 7$. Ces signes binaires, calculés pour chaque pays, mettent en évidence des motifs qualitatifs.

\subsection{Distance de Fisher-Rao}

La dissimilarité entre deux lois $\mathcal{N}(\mu_1, \sigma_1^2)$ et $\mathcal{N}(\mu_2, \sigma_2^2)$ est mesurée par la distance de Fisher-Rao, donnée par \eqref{eq:fr_normal} du chapitre \ref{chap:math}. Les longueurs totales des trajectoires le long des âges quantifient les évolutions globales : une distance plus grande indique une évolution plus marquée, \emph{ceteris paribus}.

\subsection{Divergence de Kullback-Leibler}

La dissimilarité est évaluée par la divergence de Kullback-Leibler, dirigée, définie par \eqref{eq:kl_normal} du chapitre \ref{chap:math}. Les divergences totales le long des trajectoires par âges évaluent les évolutions directionnelles : une valeur plus élevée signale une asymétrie accrue, \emph{ceteris paribus}.

\subsection{Comparaisons transnationales}

Les résultats pour les États-Unis sont comparés à ceux du Royaume-Uni, de la Norvège et de l'Italie, en évaluant les divergences dans les trajectoires hyperboliques, les motifs symboliques, la distance de Fisher-Rao et la divergence de Kullback-Leibler. Cette analyse identifie des invariants ou hétérogénéités liées à des contextes socio-économiques distincts.

\section{L'effet de l'arsenic PM$_{2.5}$}
Les approches méthodologiques reposent sur une modélisation probabiliste appliquée aux données de prévalence des maladies respiratoires chroniques issues de GBD-2021, stratifiées par région, année, âge et sexe. Elles s'inscrivent dans le cadre défini au chapitre sur les données, où chaque quadruplet $(s, t, a, g) \in \mathcal{S} \times \mathcal{T} \times \mathcal{A} \times \mathcal{G}$ est associé à une variable aléatoire $P_{s,t,a,g} \sim \mathcal{N}(\mu_{s,t,a,g}, \sigma_{s,t,a,g}^2)$ représentant la transformée logit (en base 10) du taux de prévalence. Les concentrations d'arsenic dans les PM$_{2.5}$, notées $C_s \sim \mathcal{N}(\mu_s, \sigma_s^2)$, sont issues des données de l'EPA agrégées par région $s \in \mathcal{S}$.

\subsection{Agrégation de données}
Pour chaque combinaison de région $s \in \mathcal{S}$, âge $a \in \mathcal{A}$ et sexe $g \in \mathcal{G}$, la prévalence agrégée $P_{s,a,g}$ est obtenue comme la moyenne des $P_{s,t,a,g}$ sur toutes les années $t \in \mathcal{T}$ :
\begin{eqnarray}
\mu_{s,a,g} &=& \frac{1}{n} \sum_{k=1}^n \mu_{s,t_k,a,g}, \\
\sigma_{s,a,g} &=& \frac{1}{n} \sqrt{\sum_{k=1}^n \sigma_{s,t_k,a,g}^2},
\end{eqnarray}
où $n = \#\mathcal{T}$ et les éléments de $\mathcal{T}$ sont notés : $t_1 < t_2 < ... < t_n$. Cette agrégation réduit la variabilité temporelle en combinant les moyennes et en ajustant les écarts-types pour refléter l'incertitude cumulée, fournissant une mesure stable de la prévalence régionale stratifiée par âge et sexe.

\subsection{Régression linéaire robuste}
Pour chaque paire $(a, g) \in \mathcal{A} \times \mathcal{G}$, la relation entre la concentration d'arsenic $C_s$ (représentée par $\mu_s$) et la prévalence agrégée $P_{s,a,g}$ est modélisée par régression linéaire robuste par la méthode des médianes répétées de Siegel. Deux modèles distincts sont ajustés : l'un pour la moyenne $\mu_{s,a,g}$ en fonction de $\mu_s$, et l'autre pour le logarithme (base 10) de l'écart-type $\log_{10} \sigma_{s,a,g}$ en fonction de $\mu_s$. Cette approche duale capture à la fois les effets sur le niveau central et sur la dispersion de la prévalence.

L'estimation des paramètres (pente $\hat{\beta}_{a,g}$ et ordonnée à l'origine $\hat{\alpha}_{a,g}$) repose sur des médianes itérées des pentes et intercepts partiels, assurant une robustesse aux valeurs aberrantes avec un point de rupture asymptotique de 50\,\%. Les intervalles de confiance à 95\,\% sont dérivés des quantiles empiriques obtenus par rééchantillonnage bootstrap non paramétrique ($B = 1\,000$), où les régions sont tirées avec remplacement et les régressions réexécutées.

La déviation absolue médiane (MAD) des résidus est calculée pour quantifier la dispersion robuste autour de chaque modèle ajusté, en appliquant la constante de correction pour la consistance gaussienne. Des statistiques de test (valeur V et p-valeur) évaluent la significativité des coefficients, complétant l'analyse par une mesure de l'évidence contre l'hypothèse nulle d'absence d'effet.

\subsection{Quantification géométrique des trajectoires}
Les pentes estimées $\hat{\beta}_{a,g}$ pour chaque âge $a$ et sexe $g$ sont projetées dans le demi-plan de Poincaré $\mathbb{H} = \{ z \in \mathbb{C} \mid \Im(z) > 0 \}$, par $z_{a,g} = \hat{\beta}_{a,g} + i \cdot \mathrm{MAD}_{a,g}$, où la composante imaginaire représente la dispersion robuste des résidus. Cette représentation modélise les évolutions par âge comme des trajectoires sur une variété hyperbolique, facilitant l'analyse des dynamiques.

Les dissimilarités entre tranches d'âge consécutives sont mesurées par la distance de Fisher-Rao et la divergence de Kullback-Leibler, calculées sur les distributions normales associées aux points $(\mu, \sigma)$, où $\mu = \hat{\beta}_{a,g}$ et $\sigma = \mathrm{MAD}_{a,g}$. Les sommes cumulées de ces mesures quantifient l'évolution globale des effets de l'arsenic au fil de l'âge, avec des comparaisons entre sexes pour identifier des patterns différenciés.

Par ailleurs, les régions sont ordonnées par niveau croissant de concentration d'arsenic $\mu_s$, puis regroupées en clusters de tailles variables (1, 3 ou 17 régions par groupe). Pour chaque cluster, des moyennes et écarts-types agrégés de la prévalence sont calculés, formant des trajectoires dans le plan $(\mu, \sigma)$. Ces chemins visualisent l'effet progressif de l'exposition, avec des segments reliant les points agrégés et des barres d'erreur indiquant l'incertitude.

	\chapter{Résultats}
\label{chap:results}

Ce chapitre présente les résultats empiriques obtenus à partir des analyses décrites dans le chapitre \ref{chap:methods}. Les résultats sont organisés en sections correspondant aux approches méthodologiques principales.

\section{L'effet de l'âge stratifié par le sexe}

\subsection{Agrégation de données}

Les taux d'incidence agrégés par âge et sexe, obtenus après transformation inverse logit et bootstrapping ($B = 100\,000$), sont présentés à la Figure \ref{fig:incidence-age-sexe}.

%\begin{figure}[H]
%\centering
%\includegraphics[width=0.8\textwidth]{figures/incidence_age_sexe/incidence_age_sexe.png}
%\caption{Taux d'incidence agrégés des MRC par âge et sexe aux États-Unis (1990-2021), avec intervalles de confiance à 95\,\%.}
%\label{fig:incidence-age-sexe}
%\end{figure}
image ici

\subsection{Dynamique hyperbolique}

Les trajectoires dans le demi-plan de Poincaré des paramètres ($\mu$, $\sigma$) par âge et sexe sont présentées à la Figure \ref{fig:poincare-age-sexe}.

%\begin{figure}[H]
%\centering
%\includegraphics[width=0.8\textwidth]{figures/poincare_age_sexe/poincare_age_sexe.png}
%\caption{Trajectoires hyperboliques des paramètres ($\mu$, $\sigma$) par âge et sexe.}
%\label{fig:poincare-age-sexe}
%\end{figure}
image ici

\subsection{Dynamique symbolique}

Les indicateurs directionnels $S_{a,g}^{(\mu)}$ et $S_{a,g}^{(\sigma)}$ pour chaque pays sont présentés dans le tableau suivant.

%\input{tables/symbolique_age_sexe.tex}
table ici

\subsection{Distance de Fisher-Rao}

La somme cumulée des distances de Fisher-Rao le long des trajectoires par sexe pour chaque pays est présentée dans le tableau suivant.

%\input{tables/fr_cumule_sexe.tex}
table ici

\subsection{Divergence de Kullback-Leibler}

La somme cumulée des divergences de Kullback-Leibler le long des trajectoires par sexe pour chaque pays est présentée dans le tableau suivant.

%\input{tables/kl_cumule_sexe.tex}
table ici

\subsection{Comparaisons transnationales}

Les trajectoires hyperboliques par pays et sexe sont présentées à la Figure \ref{fig:comparaison-pays-age}.

%\begin{figure}[H]
%\centering
%\includegraphics[width=0.8\textwidth]{figures/comparaison_pays_age/comparaison_pays_age.png}
%\caption{Comparaison des trajectoires hyperboliques par pays et sexe.}
%\label{fig:comparaison-pays-age}
%\end{figure}
image ici

\section{L'effet du sexe stratifié par l'âge}

\subsection{Agrégation de données}

La statistique $g$ de Cohen pour les différences sexuelles par âge est présentée à la Figure \ref{fig:cohen-g-age}.

%\begin{figure}[H]
%\centering
%\includegraphics[width=0.8\textwidth]{figures/cohen_g_age/cohen_g_age.png}
%\caption{Statistique $g$ de Cohen pour les différences sexuelles par âge, avec intervalles de confiance.}
%\label{fig:cohen-g-age}
%\end{figure}
image ici

\subsection{Dynamique hyperbolique}

La trajectoire dans le demi-plan de Poincaré des paramètres ($\mu$, $\sigma$) de $g$ par âge est présentée à la Figure \ref{fig:poincare-sexe-age}.

%\begin{figure}[H]
%\centering
%\includegraphics[width=0.8\textwidth]{figures/poincare_sexe_age/poincare_sexe_age.png}
%\caption{Trajectoire hyperbolique des paramètres de $g$ par âge.}
%\label{fig:poincare-sexe-age}
%\end{figure}
image ici

\subsection{Dynamique symbolique}

Les indicateurs directionnels $S_a^{(\mu)}$ et $S_a^{(\sigma)}$ pour chaque pays sont présentés dans le tableau suivant.

%\input{tables/symbolique_sexe_age.tex}
table ici

\subsection{Distance de Fisher-Rao}

La longueur cumulée des trajectoires de Fisher-Rao par pays est présentée dans le tableau suivant.

%\input{tables/fr_cumule_age.tex}
table ici

\subsection{Divergence de Kullback-Leibler}

La somme cumulée des divergences de Kullback-Leibler par pays est présentée dans le tableau suivant.

%\input{tables/kl_cumule_age.tex}
table ici

\subsection{Comparaisons transnationales}

La statistique $g$ de Cohen par pays et âge est présentée à la Figure \ref{fig:comparaison-pays-sexe}.

%\begin{figure}[H]
%\centering
%\includegraphics[width=0.8\textwidth]{figures/comparaison_pays_sexe/comparaison_pays_sexe.png}
%\caption{Comparaison de $g$ de Cohen par pays et âge.}
%\label{fig:comparaison-pays-sexe}
%\end{figure}
image ici

\section{L'effet de l'arsenic PM$_{2.5}$ sur la prévalence des maladies respiratoires chroniques}

\subsection{Agrégation de données}

Les paramètres agrégés de prévalence ($\mu_{s,a,g}$, $\sigma_{s,a,g}$) par région, âge et sexe sont obtenus comme décrit.

\subsection{Régression linéaire robuste}

Les estimations des pentes $\hat{\beta}_{a,g}$, ordonnées à l'origine $\hat{\alpha}_{a,g}$, intervalles de confiance, MAD des résidus et statistiques de test pour les modèles sur $\mu_{s,a,g}$ et $\log_{10} \sigma_{s,a,g}$ en fonction de $\mu_s$ sont présentées dans le tableau suivant.

%\input{tables/regression_siegel.tex}
table ici

\subsection{Quantification géométrique des trajectoires}

Les trajectoires dans le demi-plan de Poincaré des $\hat{\beta}_{a,g}$ + $i \cdot \mathrm{MAD}_{a,g}$ par âge et sexe, les sommes cumulées des distances de Fisher-Rao et divergences de Kullback-Leibler par sexe, ainsi que les trajectoires agrégées par clusters de régions ordonnées par $\mu_s$ sont présentées à la Figure \ref{fig:trajectoires-arsenic}.

%\begin{figure}[H]
%\centering
%\includegraphics[width=0.8\textwidth]{figures/trajectoires_arsenic/trajectoires_arsenic.png}
%\caption{Trajectoires des effets de l'arsenic par âge et sexe.}
%\label{fig:trajectoires-arsenic}
%\end{figure}
image ici

Les sommes cumulées des distances de Fisher-Rao et divergences de Kullback-Leibler pour les trajectoires des effets de l'arsenic sont présentées dans le tableau suivant.

%\input{tables/fr_kl_arsenic.tex}
table ici
	%\chapter{Discussion}
\label{chap:discussion}

Ce chapitre discute et synthétise les résultats présentés au chapitre \ref{chap:results}, identifie les limites de l'étude et explore leurs implications théoriques et pratiques. Les analyses, fondées sur la géométrie de l'information, portent sur les données d'incidence et de prévalence des maladies respiratoires chroniques, stratifiées par âge, sexe et exposition régionale à l'arsenic contenu dans les PM$_{2,5}$.

\section{Synthèse des résultats}

Les résultats empiriques révèlent des motifs cohérents dans les variations des maladies respiratoires chroniques en fonction de l'âge, du sexe et de l'exposition à l'arsenic PM$_{2,5}$. Dans l'analyse de l'effet de l'âge stratifié par le sexe, les intervalles de confiance des taux d'incidence (Figure \ref{fig:CI-age-sex}) indiquent une évolution en forme de U, avec des taux plus élevés chez les très jeunes et les personnes âgées, plus marquée chez les mâles que chez les femelles. Les trajectoires hyperboliques dans le demi-plan de Poincaré (Figures \ref{fig:age-sex_Poincare_US_UK} et \ref{fig:age-sex_Poincare_NO_IT}) exhibent des mouvements généralement antihoraire, reflétant des changements dans les paramètres ($\mu, \sigma$) au fil de l'âge, avec des motifs similaires à travers les pays étudiés (États-Unis, Royaume-Uni, Norvège, Italie).

Les signes des différences entre tranches d'âge consécutives pour $\mu$ et $\sigma$ (Tables \ref{tab:age-sex-symbol-mu-male}, \ref{tab:age-sex-symbol-mu-female}, \ref{tab:age-sex-symbol-sigma-male} et \ref{tab:age-sex-symbol-sigma-female}) démontrent une constance qui semble être de nature biologique plutôt qu’environnementale. Les sommes cumulées des distances de Fisher-Rao (Figure \ref{fig:age-sex_FR_cumulative}) et des divergences de Kullback-Leibler (Figure \ref{fig:age-sex_KL_cumulative}) sont systématiquement plus élevées chez les hommes, soulignant une plus grande variabilité distributionnelle masculine, observée de manière robuste à travers les pays.

Pour l'effet du sexe stratifié par l'âge, les intervalles de confiance de la statistique g de Cohen (Figure \ref{fig:sex-age_plot_country-sex}) mettent en évidence une prédominance masculine en enfance, féminine à l'adolescence et chez les jeunes adultes, puis une variabilité inter-pays à partir de 50 ans, avec une tendance à la prédominance masculine chez les âgés. Les trajectoires hyperboliques associées (Figure \ref{fig:sex-age_Poincare}) semblent être un U inversé qui oscille d'un côté à l'autre de la ligne $\mu = 0$, rappelant la trajectoire du tige de pendule d'un métronome mécanique. Les signes des changements pour $\mu$ et $\sigma$ (Tables \ref{tab:age-sex-symbol-mu} et \ref{tab:age-sex-symbol-sigma}) sont similaires entre pays, renforçant la généralisabilité.

Concernant l'effet de l'arsenic PM$_{2,5}$, les distributions géographiques (Figures \ref{fig:arsenic-map} et \ref{fig:arsenic_Poincare}) montrent des concentrations moyennes variant par État aux États-Unis, avec des paramètres ($\mu_s, \sigma_s$) indiquant des hétérogénéités régionales. Les log-cotes moyennes de prévalence des MRC (Figures \ref{fig:GBD-map} et \ref{fig:GBD_Poincare}) révèlent des variations spatiales marquées, avec des paramètres ($\mu_{s,a,g}, \sigma_{s,a,g}$) soulignant des disparités. 

Les analyses de validation et de diagnostic, centrées sur les modèles de régression linéaire des moindres carrés ordinaires pour l'effet de l'arsenic, confirment la robustesse globale des résultats. Les métriques de performance, telles que la distribution du coefficient de détermination $R^2$ (Figure~\ref{fig:global_r2}) et les tendances par âge et sexe (Figure~\ref{fig:trends_r2}), indiquent un ajustement satisfaisant, avec une proportion élevée d'associations significatives (Figure~\ref{fig:global_p}). Les tests diagnostiques, incluant le RESET pour la linéarité (Figure~\ref{fig:reset_hist}), le Shapiro-Wilk pour la normalité (Figure~\ref{fig:shapiro_hist}), le Breusch-Pagan pour l'homoscédasticité (Figure~\ref{fig:bp_hist}) et le Durbin-Watson pour l'autocorrélation (Figure~\ref{fig:dw_hist}), valident les hypothèses sous-jacentes pour la majorité des modèles, bien que des valeurs aberrantes et points influents dans certains sous-groupes (Figures~\ref{fig:outliers}, \ref{fig:leverage} et \ref{fig:cooks}) suggèrent une sensibilité limitée. La régression robuste complémentaire (méthode de Siegel), commentée ci-dessous, renforce ces conclusions, assurant la fiabilité des associations observées.

 En effet, les intervalles de confiance des pentes $\hat{\beta}_{a,g}$ (Figure \ref{fig:CI-arsenic}) indiquent une relation positive monotone avec l'exposition, modulée par l'âge et le sexe. Les trajectoires dans le demi-plan de Poincaré (Figure \ref{fig:trajectoires-arsenic}) et les distances cumulées de Fisher-Rao (Figure \ref{fig:arsenic_distance_cumulative-Fisher-Rao}) et Kullback-Leibler (Figure \ref{fig:arsenic_distance_cumulative-Kullback-Leibler}) exhibent une variabilité accrue chez les hommes, suggérant une sensibilité sexuelle différentiée à l'exposition.

Par conséquent, la régression linéaire (ordinaire ou robuste) fournit un bon modèle explicatif de l’effet de l’exposition à l’arsenic PM$_{2.5}$ sur la prévalence des maladies respiratoires chroniques, mais ce modèle trop simple est faible pour faire des prédictions.

\section{Limites de l'étude}

Des facteurs confondants potentiels — tabagisme, exposition à d'autres polluants atmosphériques (par exemple ozone, dioxyde d'azote) ou déterminants socio-économiques — n'ont pas été contrôlés et pourraient influencer les associations observées. 

L'utilisation de concentrations régionales d'arsenic dans les PM$_{2,5}$ comme proxy d'exposition présente des limites : ces mesures agrégées n'ont pas de résolution individuelle, et bien que des effets synergiques avec les PM$_{2,5}$ soient documentés, la voie d'exposition principale de l'arsenic reste souvent orale plutôt qu'inhalatoire.

De plus, les bases de données utilisées (GBD-2021 et EPA), couvrant la période 1990–2019, comportent des incertitudes de modélisation et une résolution spatiale limitée. L'approche par bootstrap employée pour estimer la variabilité suppose une indépendance qui pourrait être violée en présence de corrélations temporelles ou spatiales non modélisées. Enfin, cette décrit des corrélations et ne permet pas d'établir des relations causales.

\section{Implications théoriques}

Ces résultats s'inscrivent dans le cadre de la théorie évolutive de Geodakyan, qui postule une plus grande variabilité phénotypique chez les mâles en faveur d'une adaptation environnementale, tandis que les femelles conserveraient une stabilité phylogénétique. La variabilité accrue observée chez les hommes dans les trajectoires distributionnelles (Figures \ref{fig:age-sex_FR_cumulative} et \ref{fig:age-sex_KL_cumulative}), cohérente entre pays, apporte un soutien empirique quantitatif à cette perspective via la géométrie de l'information. Ces éléments contribuent au débat sur les fondements biologiques de la théorie et suggèrent d'éventuelles extensions à d'autres traits adaptatifs.

\section{Implications pratiques}

Sur le plan opérationnel, ces résultats ont des implications pour l'actuariat : les modèles de tarification des primes d'assurance santé pourraient intégrer les interactions âge–sexe–pollution afin d'affiner l'évaluation des risques au niveau régional. En épidémiologie, les méthodes fondées sur les distances de Fisher–Rao et les divergences de Kullback–Leibler peuvent être transposées à d'autres pathologies pour quantifier les différences sexuelles, et potentiellement informées a priori par la théorie de Geodakyan.

En sciences de l'environnement, la corrélation observée entre pollution atmosphérique et prévalence des MRC, associée à des indications selon lesquelles la sécheresse peut accroître la mobilisation de l'arsenic dans les sols et l'eau, souligne la nécessité d'étudier les impacts indirects du changement climatique sur la santé respiratoire. En santé publique et en politiques publiques, ces résultats plaident pour une allocation des ressources sanitaires mieux ciblée, fondée sur des données d'âge, de sexe et de niveaux régionaux de pollution plutôt que sur une distribution uniforme.

Ces implications motivent des recherches futures, notamment des études longitudinales et des approches permettant un meilleur contrôle des confondeurs, afin d'affiner les modèles et de confirmer les associations observées dans des contextes plus variés.

	%\chapter{Conclusion et perspectives}
\label{chap:conclusion}

Ce chapitre synthétise les principaux résultats de ce mémoire, met en lumière ses contributions et propose des perspectives pour des travaux futurs. Les analyses effectuées, fondées sur la géométrie de l'information et appliquées aux données d'incidence et de prévalence des maladies respiratoires chroniques, ont révélé des patterns robustes liés à l'âge, au sexe et à l'exposition à l'arsenic dans les PM$_{2,5}$.

\section{Synthèse des résultats principaux}

Les résultats empiriques démontrent une variabilité plus élevée chez les hommes que chez les femmes dans les trajectoires des paramètres de distribution des incidences des MRC. Spécifiquement, la somme cumulée des distances de Fisher-Rao entre trajectoires consécutives est systématiquement supérieure chez les hommes, une propriété vérifiée sur des données provenant de plusieurs pays (Figures~\ref{fig:age-sex_FR_cumulative} et~\ref{fig:sex-age_distance_cumulative-Fisher-Rao}). Un résultat analogue émerge avec la divergence de Kullback-Leibler (Figures~\ref{fig:age-sex_KL_cumulative} et~\ref{fig:sex-age_distance_cumulative-Kullback-Leibler}), suggérant que ces mesures capturent une caractéristique biologique intrinsèque des populations.

Ces observations s'interprètent à travers la théorie évolutive de Geodakyan, qui postule une plus grande variabilité phénotypique chez les mâles pour favoriser l'adaptation environnementale. La taille d'effet mesurée par la statistique $g$ de Cohen révèle une prédominance masculine pendant l'enfance, féminine à l'adolescence et chez les jeunes adultes, puis une variabilité selon les pays à partir de 50 ans, avec une tendance générale vers une prédominance masculine (Figure~\ref{fig:sex-age_plot_country-sex}). Les variations par tranche d'âge, codées en signes d'augmentation ou de diminution pour $\mu$ et $\sigma$, sont consistantes entre pays (Tables~\ref{tab:age-sex-symbol-mu-male} à~\ref{tab:age-sex-symbol-sigma-female} et Tables~\ref{tab:age-sex-symbol-mu} à~\ref{tab:age-sex-symbol-sigma}).

Concernant l'effet de l'arsenic dans les PM$_{2,5}$, la relation avec la prévalence augmente linéairement avec l'exposition après transformations\footnote{La transformation log pour la concentration moyenne d'arsenic PM$_{2,5}$ et la transformation log-cotes pour la prévalence.}, confirmant un lien causal sous-jacent (Figures~\ref{fig:arsenic-map}, \ref{fig:arsenic_Poincare}, \ref{fig:GBD-map} et~\ref{fig:GBD_Poincare}.

\section{Contributions}

Ce mémoire contribue à l'intégration de la géométrie de l'information en épidémiologie, en démontrant l'utilité des distances de Fisher-Rao et des divergences de Kullback-Leibler pour quantifier les différences sexuelles et environnementales dans les MRC. Il renforce empiriquement la théorie de Geodakyan en fournissant des preuves quantitatives cohérentes entre pays. De plus, il met en évidence le rôle modulateur de l'arsenic aérien, en distinguant incidence et prévalence pour corriger les biais potentiels.

\section{Perspectives}

Les implications pratiques ouvrent plusieurs avenues. En actuariat, les modèles de tarification des primes d'assurance santé pourraient incorporer ces interactions âge-sexe-pollution pour une évaluation plus précise des risques régionaux. En épidémiologie, les méthodes proposées s'étendent à d'autres pathologies, contribuant au débat sur la validité de la théorie de Geodakyan via des prédictions a priori de variabilité accrue chez les hommes.

Sur le plan environnemental, les liens avec le changement climatique – notamment l'augmentation des concentrations d'arsenic due à la sécheresse – motivent des recherches sur les impacts indirects sur la santé humaine. En sciences politiques et santé publique, ces résultats plaident pour une allocation ciblée des ressources, priorisant les groupes vulnérables basés sur des données empiriques plutôt qu'une répartition uniforme.

Des travaux futurs pourraient intégrer d'autres polluants, utiliser des données longitudinales individuelles pour affiner les proxies d'exposition, ou appliquer ces approches à des contextes globaux pour valider la généralisabilité.
	
	
	
	% -------------------------------------------------
	% Annexes
	% -------------------------------------------------
	\appendix
	%\chapter{Figures et tables complémentaires}

	
	\bigskip
	\bibliographystyle{plainnat}
	\bibliography{biblio}

\end{document}