\PassOptionsToPackage{numbers,sort&compress}{natbib}
\documentclass[MSc,french]{ulthese}
\usepackage{ragged2e} % \RaggedRight, \Centering, etc.
% ----------------------------------------------------------------------
% ENCODAGE, LANGUE & MICRO‑TYPOGRAPHIE

% ----------------------------------------------------------------------
\usepackage[T1]{fontenc} % codage de fonte moderne
\usepackage[utf8]{inputenc} % encodage de la source (pdfLaTeX uniquement)
\usepackage{csquotes} % guillemets « français »/‹ anglais › …
\usepackage{microtype} % protrusion & expansion
%

%----------------------------------------------------------------------
% POLICES
% ----------------------------------------------------------------------
\usepackage{lmodern} % Latin Modern — vectoriel + T1
% ----------------------------------------------------------------------
% MATHÉMATIQUES

% ----------------------------------------------------------------------
\usepackage{amsmath,amssymb,amsthm}
\usepackage{icomma} % espace fine après virgule decimale
% \usepackage{mathtools} % décommentez pour besoins avancés
% ----------------------------------------------------------------------
% TABLEAUX & NOMBRES
% ----------------------------------------------------------------------

\usepackage{booktabs,longtable,tabularx}
\usepackage{siunitx}
\sisetup{
	output-decimal-marker = {,},
	detect-all, % détecte famille/gras/italique
	per-mode = symbol % \si{\per\something}
}
% ----------------------------------------------------------------------
% GRAPHIQUES & FIGURES
% ----------------------------------------------------------------------
\usepackage{graphicx}
\usepackage{enumitem}
\usepackage{subcaption}
\usepackage{float} % placement précis [H]
\usepackage{placeins} % \FloatBarrier
% PGF/TikZ & PGFPLOTS
\usepackage{tikz}
\usetikzlibrary{arrows.meta,positioning,shapes.geometric}
\usepackage{pgfplots}
\pgfplotsset{compat=1.18}
\usepackage{tikz-cd}
% ----------------------------------------------------------------------
% BIBLIOGRAPHIE
% ----------------------------------------------------------------------
\usepackage{chapterbib} % bibliographies par chapitre (nécessite \bibliography)
% natbib est déjà chargé par la classe ulthese avec nos options ci‑dessus
% ----------------------------------------------------------------------
% HYPERLIENS & RÉFÉRENCES CROISÉES
% ----------------------------------------------------------------------
\usepackage{xcolor}
\usepackage[unicode,colorlinks=true,linkcolor=teal,citecolor=teal,urlcolor=teal,
bookmarksopen=true,bookmarksdepth=3]{hyperref}
\usepackage{cleveref} % après hyperref
% ----------------------------------------------------------------------
% UTILITAIRES
% ----------------------------------------------------------------------
\usepackage{etoolbox}

\usepackage{listings}
\lstset{
	language=R,
	basicstyle=\ttfamily\small,
	keywordstyle=\color{blue}\bfseries,
	stringstyle=\color{red},
	commentstyle=\color{green!60!black}\itshape,
	numbers=left,
	numberstyle=\tiny\color{gray},
	stepnumber=1,
	numbersep=5pt,
	backgroundcolor=\color{gray!10},
	showstringspaces=false,
	frame=single,
	tabsize=2,
	breaklines=false,
	breakatwhitespace=false,
	captionpos=b,
	escapeinside={\%*}{*)},
	morekeywords={library, here, fs, janitor, define_constants_gbd, generate_file_paths, generate_output_path, compute_logit_1e5, process_gbd_data, map_dfr, clean_names, filter, mutate, select, distinct, write_csv}
}



% ----------------------------------------------------------------------
% PATCHES
% ----------------------------------------------------------------------
\patchcmd{\chapter}{\par}{\par\vspace{1em}}{}{}
% ----------------------------------------------------------------------
% CONFIGURATION CSV (optionnel)
% ----------------------------------------------------------------------
\usepackage{csvsimple}
\csvset{
	separator = {comma},
	respect all,
	tabular = {r},
}
% ----------------------------------------------------------------------
% ENVIRONNEMENTS DE THÉORÈMES (noms français)
% ----------------------------------------------------------------------
\theoremstyle{definition}
\newtheorem{theorem}{Théorème}
\newtheorem{conj}{Conjecture}
\newtheorem{lemma}[theorem]{Lemme}
\newtheorem{corollary}[theorem]{Corollaire}
\newtheorem{proposition}[theorem]{Proposition}
\newtheorem{definition}{Définition}
\newtheorem{example}[theorem]{Exemple}
\newtheorem{remark}{Remarque}
\newtheorem{notation}[theorem]{Notation}
% ----------------------------------------------------------------------
% MACROS & NOTATIONS PERSONNALISÉES


% ----------------------------------------------------------------------


% Extensions graphiques (pdfLaTeX)
\DeclareGraphicsExtensions{.pdf,.png,.jpg,.eps}
% ----------------------------------------------------------------------
% MÉTADONNÉES PDF & PAGE TITRE
% ----------------------------------------------------------------------
\hypersetup{
	pdftitle = {Application de la géométrie de l’information à l’étude épidémiologique des maladies respiratoires chroniques},
	pdfauthor = {José Manuel Rodríguez Caballero}
}
% Info de page‑titre (macros de la classe)
\titre{Application de la géométrie de l’information à l’étude épidémiologique des maladies respiratoires chroniques}
\soustitre{L'effet de la toxicité atmosphérique}
\auteur{José Manuel Rodríguez Caballero}
\programme{Maîtrise en statistique}
\direction{M'Hamed Lajmi Lakhal Chaieb}
\codirection{Karim Barigou}
\annee{2025}
% ----------------------------------------------------------------------
% DOCUMENT
% ----------------------------------------------------------------------
\begin{document}
	\frontmatter % pages liminaires
	\frontispice % page de couverture officielle
	\section{Résumé}

Ce mémoire étudie les différences liées au sexe dans les trajectoires de paramètres statistiques et la relation entre exposition atmosphérique (PM2.5 — proxy arsenic) et maladies respiratoires chroniques, en combinant méthodes géométriques de l’information, indices d’effet et analyses épidémiologiques sur jeux de données multi-pays.

\paragraph{Méthodes.} Pour quantifier les variations temporelles des paramètres $(\mu,\sigma)$ j’utilise la distance de Fisher–Rao appliquée à des familles paramétriques, complétée par la divergence de Kullback–Leibler comme mesure alternative. Les tailles d’effet (g de Cohen) sont calculées par tranche d’âge ; les relations exposition–prévalence sont modélisées par régressions linéaires appliquées à des transformations log et logit de la prévalence. Les diagnostics incluent les tests de normalité de Shapiro–Wilk sur les résidus et des ajustements pour comparaisons multiples.

\paragraph{Résultats principaux.} La somme cumulée des distances consécutives Fisher–Rao est systématiquement plus élevée chez les hommes que chez les femmes, propriété confirmée sur plusieurs pays ; observation analogue obtenue avec la divergence KL. L’analyse des tailles d’effet (g de Cohen) montre une prédominance masculine durant l’enfance, une prédominance féminine à l’adolescence et chez les jeunes adultes, puis une variabilité inter-pays après 50 ans avec une tendance générale vers une prédominance masculine. Pour l’exposition à l’arsenic (PM2.5), l’incidence varie avec l’âge : autour de 47 ans (±2) l’incidence augmente avec la concentration d’arsenic chez les hommes, tandis que chez des hommes très âgés (\~92 ±2 ans) la relation s’inverse — un biais expliqué par un effet d’accumulation et de diagnostic antérieur. En revanche, la prévalence augmente de manière robuste avec la pollution ; la relation prévalence–pollution apparaît linéaire après transformations log / logit. Les diagnostics de normalité des résidus sont très satisfaisants, sous réserve d’une correction pour tests multiples.

\paragraph{Interprétation et implications.} J’interprète ces motifs empiriques à la lumière de la théorie biologique de Geodakyan, qui fournit un cadre pour expliquer les différences sexuelles observées. Les résultats ont des implications pratiques pour l’actuariat (tarification liée à âge/sex/pollution), l’épidémiologie (méthodes transférables à d’autres pathologies) et les politiques de santé environnementale (allocation ciblée des ressources). Les données et les scripts R sont fournis en annexe. Des travaux futurs pourraient étendre l’étude à d’autres polluants et mettre en œuvre approches causales longitudinales.

	\section{Abstract}

This thesis investigates the impact of regional concentrations of arsenic in fine particulate matter (PM${2.5}$) on the incidence and prevalence of chronic respiratory diseases (CRDs), accounting for variations by age and sex. Motivated by established links between arsenic exposure and respiratory conditions, as well as synergistic effects with PM${2.5}$, the study includes a preliminary analysis of age-sex interactions without considering arsenic, interpreted through Geodakyan’s evolutionary theory, which highlights greater variability in males as a mechanism for environmental adaptation.
The implications extend to actuarial science, for refining health insurance premium models, and to public health, for targeted resource allocation.
Empirically, the analyses reveal that the cumulative sum of Fisher-Rao distances between consecutive trajectories of incidence distribution parameters is consistently higher in men than in women, with similar patterns observed using Kullback-Leibler divergence; these findings are consistent across countries. The relationship between arsenic exposure and incidence is age-dependent, with a reversal in very elderly subjects due to diagnostic bias, while prevalence increases linearly after appropriate transformations. Methodologically, the approach relies on information geometry (Fisher-Rao distance, Kullback-Leibler divergence), effect size via Cohen’s g for proportions, and bootstrapping to estimate variability.
Interpreted within Geodakyan’s theoretical framework, these results contribute to the debate on its biological foundations and open practical perspectives in actuarial science (premium adjustment), epidemiology (extension to other diseases), environmental science (links to climate change increasing airborne arsenic concentrations via drought), and political science (optimized distribution of health resources). The thesis is structured into chapters covering mathematical preliminaries, data, methodology, results, discussion, conclusions, and methodological appendices.
	\tableofcontents
	\cleardoublepage
	\listoftables
	\cleardoublepage
	\listoffigures
	\cleardoublepage
	\section*{Remerciements}
Je remercie chaleureusement mon directeur, M’Hamed Lajmi Lakhal Chaieb, et mon codirecteur, Karim Barigou, pour leur encadrement, leurs conseils et leur disponibilité tout au long de ce mémoire.

Je tiens également à exprimer ma gratitude à Monsieur Charles Brindamour, donateur de la Bourse d’excellence Charles Brindamour, dont le soutien financier et la reconnaissance académique ont grandement contribué à la réalisation de ce travail.

Je suis reconnaissant envers mes collègues de maîtrise pour leurs échanges enrichissants, ainsi qu’à ma famille et mes proches pour leur soutien constant et leurs encouragements.
	\section*{Avant-propos}
Après avoir déménagé à la ville de Québec pour poursuivre mes études à l’Université Laval, j’ai observé que, particulièrement durant l’été, la qualité de l’air était vraiment mauvaise et dangereuse pour la santé. Cette constatation m’a amené à me poser la question suivante : l’effet de la pollution atmosphérique sur la santé est-il négligeable ou non ? Ce mémoire offre une réponse partielle à cette interrogation générale. J’espère que ma contribution favorisera une prise de conscience écologique accrue et stimulera une volonté politique visant à réduire la pollution atmosphérique. Dans les situations où celle-ci s’avère inévitable, telles que les incendies naturels, il convient d’adopter des mesures pour limiter l’exposition humaine, par exemple en recourant à des masques et à des purificateurs d’air. De plus, compte tenu de l’effet du changement climatique sur la concentration atmosphérique d’arsenic à cause de la sécheresse, j’espère que cette thèse motivera la recherche sur les effets directs et indirects du changement climatique sur la santé humaine.
	\mainmatter % corps du document
	\chapter{Introduction}
\label{chap:intro}
\section{Contexte et motivation}
Présentez la question scientifique : différences liées au sexe, rôle de la pollution atmosphérique, enjeux pour l'actuariat et la santé publique.


\section{Contributions}
Récapitulatif des contributions principales (observations empiriques, approche méthodologique, interprétation selon Geodakyan, implications).


\section{Plan du mémoire}
Courte description des chapitres.
	\chapter{Préliminaires mathématiques}
\label{chap:math}
Ce chapitre présente brièvement les outils mathématiques utilisés dans les développements ultérieurs : la divergence de Kullback–Leibler, la géométrie de l'information (métrique de Fisher et distance de Fisher–Rao), ainsi que quelques repères sur la notion de taille d'effet.

\section{Divergence de Kullback–Leibler}
Pour deux lois de probabilité $P$ et $Q$ ayant respectivement pour densités $p$ et $q$, la divergence de Kullback–Leibler est définie par \citep{ay2017information} :
\begin{equation}
	D_{KL}(P\|Q) = \int p(x)\log\frac{p(x)}{q(x)}\,dx.
\end{equation}
Quelques propriétés fondamentales :
\begin{enumerate}
	\item $D_{KL}(P\|Q) \ge 0$ ;
	\item $D_{KL}(P\|Q) = 0$ si et seulement si $P = Q$ presque partout ;
	\item La divergence est asymétrique : $D_{KL}(P\|Q) = D_{KL}(Q\|P)$ n'est pas toujours vraie.
\end{enumerate}
Dans le cas de lois normales univariées $P = \mathcal{N}(\mu_1, \sigma_1^2)$ et $Q = \mathcal{N}(\mu_2, \sigma_2^2)$, on a \citep{belov2011distributions} :
\begin{equation}
	D_{\mathrm{KL}}(\mu_{1},\sigma_{1};\mu_{2},\sigma_{2})
	= \ln\left(\frac{\sigma_{2}}{\sigma_{1}}\right)
	+ \frac{\sigma_{1}^{2}+(\mu_{2}-\mu_{1})^{2}}{2\sigma_{2}^{2}} - \tfrac{1}{2}.
	\label{eq:kl_normal}
\end{equation}

\section{Géométrie de l'information}
La géométrie de l'information traite les familles de distributions de probabilité comme des variétés riemanniennes, en utilisant des outils de la géométrie différentielle pour étudier leurs propriétés \citep{ay2017information}.

\subsection{Métrique de Fisher}
Pour une famille paramétrique de densités $p(x;\theta)$, la matrice d'information de Fisher, qui définit la métrique riemannienne, est donnée par :
\begin{equation}
	g_{ij}(\theta) = \mathbb{E}_{\theta}\!\left[\partial_{i}\log p(X;\theta)\,\partial_{j}\log p(X;\theta)\right].
\end{equation}

\subsection{Distance de Fisher–Rao}
La distance de Fisher–Rao $d_{\mathrm{FR}}(\theta_1,\theta_2)$ est la longueur minimale d’une courbe reliant $\theta_1$ à $\theta_2$ selon la métrique de Fisher.

Pour deux lois normales univariées $\mathcal{N}(\mu_1, \sigma_1^2)$ et $\mathcal{N}(\mu_2, \sigma_2^2)$, la distance de Fisher–Rao est \citep{nielsen2023simple} :
\begin{equation}
	d_{\mathrm{FR}}\big((\mu_1,\sigma_1),(\mu_2,\sigma_2)\big)
	= \sqrt{2}\,\operatorname{acosh}\!\left(1 + \frac{(\mu_2 - \mu_1)^2 + 2(\sigma_2 - \sigma_1)^2}{4\sigma_1\sigma_2}\right).
	\label{eq:fr_normal}
\end{equation}

\section{Taille de l'effet}
La taille de l'effet quantifie l'ampleur d'une différence indépendamment de la taille de l’échantillon. Elle est utile pour interpréter la pertinence pratique d’un résultat statistique.

\subsection{Proportions (g de Cohen)}
Pour une proportion $P$, la taille d’effet selon Cohen est définie par :
\begin{equation}
	g = P - \tfrac{1}{2}.
\end{equation}
Selon les seuils proposés par Jacob Cohen \citep{cohen1992power}, on considère :
\begin{enumerate}
	\item $|g| \approx 0{,}05$ : effet faible ;
	\item $|g| \approx 0{,}15$ : effet moyen ;
	\item $|g| \approx 0{,}25$ : effet fort.
\end{enumerate}

\section{Bootstrapping}
Le bootstrapping est une méthode de rééchantillonnage statistique qui permet d'estimer la variabilité d'un estimateur ou de construire des intervalles de confiance à partir d'un seul échantillon, sans supposer une forme paramétrique spécifique pour la distribution sous-jacente \citep{tibshirani1993introduction}.

L'approche non paramétrique consiste à générer de multiples échantillons bootstrap en tirant avec remplacement $n$ observations à partir de l'échantillon original de taille $n$. Pour une statistique d'intérêt $\hat{\theta}$, on calcule $\hat{\theta}^*$ pour chaque échantillon bootstrap. La distribution empirique des $\hat{\theta}^*$ approxime la distribution d'échantillonnage de $\hat{\theta}$.

Par exemple, l'écart-type bootstrap d'un estimateur est :
\begin{equation}
	\widehat{\mathrm{SE}}_{\mathrm{boot}} = \sqrt{\frac{1}{B-1} \sum_{b=1}^B (\hat{\theta}^{*b} - \bar{\hat{\theta}}^*)^2},
\end{equation}
où $B$ est le nombre d'échantillons bootstrap, $\hat{\theta}^{*b}$ est la statistique pour le $b$-ième bootstrap, et $\bar{\hat{\theta}}^*$ est la moyenne des $\hat{\theta}^{*b}$.

Des variantes incluent le bootstrap paramétrique, où les rééchantillons sont tirés d'une distribution ajustée à l'échantillon original.
	\chapter{Présentation et nettoyage des données}
\label{chap:data}

\section{Données de l'EPA}

\subsection{Données brutes et nettoyage}

Les données brutes, issues de l'Environmental Protection Agency (EPA) \citep{epa_air_quality_system_2024} des États-Unis, consistent en des mesures quotidiennes de concentrations d'arsenic dans les particules fines PM$_{2,5}$, collectées de 1990 à 2019. Chaque fichier annuel, tel que \texttt{daily\_HAPS\_2017.csv}, inclut des enregistrements avec des identifiants géographiques (codes d'État, de comté et de site), des coordonnées spatiales, le paramètre mesuré (« Arsenic PM2.5 LC »), la durée d'échantillonnage, la date locale, l'unité de mesure (microgrammes par mètre cube, notée LC pour « local conditions ») et la valeur maximale observée sur 24 heures. Ces données exhibent une variabilité due aux conditions météorologiques, aux sources d'émission locales et aux protocoles de mesure.

Le nettoyage exclut les régions comme Puerto Rico, les Îles Vierges et le Mexique, pour se concentrer sur les 50 États des États-Unis et le District of Columbia, compatibles avec les données de santé publique. Les observations sont filtrées pour retenir uniquement le paramètre « Arsenic PM2.5 LC » en microgrammes par mètre cube sur 24 heures. Le nom « District Of Columbia » est corrigé en « District of Columbia » pour uniformité.

Les données annuelles sont concaténées en un ensemble unique, avec renommage des variables : région (nom de l'État ou DC) et valeur (concentration maximale quotidienne).

Pour estimer la moyenne des concentrations par région, une approche bootstrap non paramétrique est appliquée. L'ensemble des données (observations quotidiennes de 1990 à 2019) est rééchantillonné avec remplacement 1\,000 fois ($B = 1\,000$), en utilisant des graines pseudo-aléatoires pour la reproductibilité. Pour chaque rééchantillon, les valeurs sont groupées par région et le logarithme (en base 10) de la moyenne arithmétique est calculé. Pour chaque région, l'estimateur du paramètre $\mu$ est la moyenne des 1\,000 logarithmes des moyennes bootstrap, offrant un estimateur ponctuel robuste de la concentration typique, atténuant la variabilité temporelle et les valeurs aberrantes. L'écart-type des logarithmes des moyennes bootstrap fournit une estimation du paramètre $\sigma$. Le test de normalité de Shapiro-Wilk est appliqué à la distribution des logarithmes des moyennes bootstrap pour évaluer la normalité de la distribution d'échantillonnage.

Ces résultats par région ($\mu$, $\sigma$, le nombre d'échantillons bootstrap $B$, statistique $W$ et valeur $p$ de Shapiro-Wilk) sont enregistrés dans le fichier :
\begin{center}
	\texttt{arsenic\_pm2.5\_locations.csv}
\end{center}

\subsection{Données nettoyées}

Soit $\mathcal{S}$ l'ensemble des régions d'un pays donné. Pour les États-Unis, chaque région est un État ou DC. Pour chaque région $s \in \mathcal{S}$, on considère la variable aléatoire $C_s \sim \mathcal{N}(\mu_s, \sigma_s^2)$, représentant le logarithme (en base 10) de la moyenne arithmétique de la concentration quotidienne d'arsenic PM$_{2,5}$. Les paramètres $\mu_s$ sont visualisés sur la carte de la Figure \ref{fig:arsenic-map}, qui illustre les concentrations moyennes par État. Les paramètres $(\mu_s, \sigma_s)$ sont représentés dans la Figure \ref{fig:arsenic_Poincare}.

\begin{figure}[H]
	\centering
	\includegraphics[width=1\textwidth]{figures/arsenic_map/arsenic_map.png}
	\caption{Concentrations moyennes d'arsenic PM$_{2.5}$ par État aux États-Unis (1990-2019).}
	\label{fig:arsenic-map}
\end{figure}

\begin{figure}[H]
	\centering
	\includegraphics[width=1\textwidth]{figures/arsenic_Poincare/arsenic_Poincare.png}
	\caption{Paramètres $(\mu, \sigma)$ par État aux États-Unis (1990-2019).}
	\label{fig:arsenic_Poincare}
\end{figure}

\section{Données du GBD-2021}
\label{sec:gbd}

L'incidence est le nombre de nouveaux cas pour 100\,000 éléments de la population satisfaisant des caractéristiques données. La prévalence est le nombre de cas existants pour 100\,000 éléments de la population satisfaisant des caractéristiques données \citep{rothman2024epidemiology, kleinbaum1991epidemiologic}.

\subsection{Données brutes et nettoyage}

Les données brutes proviennent de l'étude Global Burden of Disease (GBD) 2021 de l'Institute for Health Metrics and Evaluation (IHME) \citep{IHME2024GBD}, et consistent en des estimations d'incidence par 100\,000 habitants, avec bornes inférieures et supérieures d'intervalles de confiance à 95\,\%, pour les États-Unis. Ces estimations sont stratifiées par région (États ou district), année (de 1990 à 2021), tranche d'âge (chacune contenant 5 ans) et sexe (mâle et femelle). Les données sont réparties sur plusieurs fichiers CSV, chacun couvrant une partie des estimations.

Le nettoyage commence par la lecture et la concaténation des fichiers pour le pays « US ». Les tranches d'âge catégorielles sont associées à leurs points moyens (par exemple, « <5 years » à 2, « 5-9 years » à 7, etc.) pour obtenir des valeurs entières. Les années sont converties en entiers.

Pour modéliser l'incidence, une transformation log-odds en base 10 est appliquée aux bornes inférieures et supérieures : $\log_{10} \left( \frac{x}{10^5 - x} \right)$, où $x$ est la valeur par 100\,000. Sur cette échelle transformée, la moyenne $\mu$ est estimée comme la moyenne des bornes transformées, et l'écart-type $\sigma$ comme $(U - L) / (2 \times z_{0,975})$, où $U$ et $L$ sont les bornes transformées, en supposant un intervalle de confiance à 95\,\% pour une distribution normale et $z_{0,975} \approx 1{,}96$ est le quantile $0,975$ de la distribution normale standard.

Les colonnes finales incluent le pays, la région, l'année, l'âge (point moyen d'une tranche de 5 ans), le sexe, $\mu$ et $\sigma$. L'ensemble nettoyé est enregistré dans le fichier :
\begin{center}
	\texttt{IHME-GBD\_2021\_CLEAN\_incidence.csv}
\end{center}

\subsection{Données nettoyées}

Soient $\mathcal{T}$, $\mathcal{A}$ et $\mathcal{G}$ les ensembles des années, tranches d'âges et sexes, respectivement. Pour chaque combinaison de région $s \in \mathcal{S}$, année $t \in \mathcal{T}$, âge $a \in \mathcal{A}$ et sexe $g \in \mathcal{G}$ aux États-Unis, la variable aléatoire transformée $I_{s,t,a,g}$ (log-cotes de l'incidence) suit une distribution $\mathcal{N}(\mu_{s,t,a,g}, \sigma_{s,t,a,g}^2)$.
	\chapter{Méthodes}
\label{chap:methods}
Ce chapitre détaille les approches méthodologiques employées pour analyser les données sur les maladies respiratoires chroniques (MRC), issues de la base GBD-2021. Les méthodes sont organisées par effet étudié : l'âge stratifié par sexe, le sexe stratifié par âge, et l'exposition à l'arsenic dans les PM$_{2.5}$. Elles reposent sur un cadre probabiliste où les taux transformés en logit (base 10) sont modélisés par des distributions normales, facilitant l'application de techniques géométriques et statistiques robustes.

\section{L'effet de l'âge stratifié par le sexe}

Cette section détaille les méthodes employées pour évaluer l'impact de l'âge sur les taux d'incidence des maladies respiratoires chroniques (MRC), stratifiés par sexe. Les analyses reposent sur des agrégations bootstrap des données GBD-2021, des projections dans l'espace hyperbolique et des mesures de dissimilarité géométrique, appliquées par pays lorsque pertinent.

\subsection{Agrégation de données}

Pour chaque paire $(a, g) \in \mathcal{A} \times \mathcal{G}$, avec $\mathcal{A}$ l'ensemble des tranches d'âge et $\mathcal{G} = \{\text{mâle}, \text{femelle}\}$, un bootstrap non paramétrique est effectué avec $B = 1\,000$ rééchantillons de taille $n = \#(\mathcal{S} \times \mathcal{T})$, où $\mathcal{S}$ désigne les emplacements (pays ou régions) et $\mathcal{T}$ les années (1990-2019). Pour le $b$-ième rééchantillon, la moyenne est calculée comme suit :
 \begin{equation}
I_{a,g}^{(b)} = \frac{1}{n} \sum_{i=1}^n I_{s_i^{(b)}, t_i^{(b)}, a, g}^{(b,i)},
 \end{equation}
où $I_{s_i^{(b)}, t_i^{(b)}, a, g}^{(b,i)} \sim \mathcal{N}(\mu_{s_i^{(b)}, t_i^{(b)}, a, g}, \sigma_{s_i^{(b)}, t_i^{(b)}, a, g}^2)$ représente la transformée logit (base 10) du taux d'incidence. Les paramètres $\mu_{a,g}$ et $\sigma_{a,g}$ sont estimés comme la moyenne et l'écart-type des $I_{a,g}^{(b)}$. Le taux d'incidence moyen $i_{a,g}$ est obtenu par inversion :
 \begin{equation}
i_{a,g} = \frac{10^5}{1 + 10^{-\mu_{a,g}}},
 \end{equation}
avec des intervalles de confiance à 95\,\% dérivés des quantiles empiriques (2,5\,\% et 97,5\,\%) des valeurs transformées, assurant une quantification robuste de l'incertitude sur l'échelle des taux par 100\,000 habitants.

\subsection{Dynamique hyperbolique}

Les paires $(\mu_{a,g}, \sigma_{a,g})$ sont projetées dans le demi-plan de Poincaré $\mathbb{H} = \{ z \in \mathbb{C} \mid \Im(z) > 0 \}$ via $z_{a,g} = \mu_{a,g} + i \sigma_{a,g}$. Cette représentation exploite la structure riemannienne de l'espace des distributions normales univariées, modélisant les variations par âge comme des trajectoires sur une variété hyperbolique, avec l'âge jouant le rôle de paramètre temporel.

\subsection{Dynamique symbolique}

Les variations qualitatives sont analysées via les signes des différences entre tranches d'âge consécutives (espacées de 5 ans) :
 \begin{equation}
S_{a,g}^{(\mu)} = \mathrm{sign} (\mu_{a,g} - \mu_{a-5,g}), \quad S_{a,g}^{(\sigma)} = \mathrm{sign}(\sigma_{a,g} - \sigma_{a-5,g}),
 \end{equation}
pour $a \geq 5$ (en tenant compte des tranches commençant à 0-4 ans). Ces indicateurs, calculés et agrégés par pays, mettent en évidence des motifs directionnels dans l'évolution des paramètres $\mu$ et $\sigma$.

\subsection{Distance de Fisher-Rao}

La distance de Fisher-Rao entre deux distributions $\mathcal{N}(\mu_1, \sigma_1^2)$ et $\mathcal{N}(\mu_2, \sigma_2^2)$ est donnée par l'équation~\eqref{eq:fr_normal} du chapitre~\ref{chap:math}. Les sommes cumulées de ces distances le long des trajectoires par âge et sexe quantifient l'ampleur globale des évolutions distributionnelles, une valeur plus élevée indiquant des changements plus marqués.

\subsection{Divergence de Kullback-Leibler}

La divergence de Kullback-Leibler, mesure asymétrique de dissimilarité, est définie par l'équation~\eqref{eq:kl_normal} du chapitre~\ref{chap:math}. Ses sommes cumulées le long des trajectoires évaluent la directionnalité des changements, une valeur plus élevée signalant une asymétrie accrue dans les transitions distributionnelles.

\section{L'effet du sexe stratifié par l'âge}

Cette section détaille les méthodes employées pour évaluer l'impact du sexe sur les taux d'incidence des maladies respiratoires chroniques (MRC), stratifiés par âge. Les analyses reposent sur des agrégations bootstrap des données GBD-2021, des projections dans l'espace hyperbolique et des mesures de dissimilarité géométrique, appliquées par pays lorsque pertinent.

\subsection{Agrégation de données}

Pour chaque âge $a \in \mathcal{A}$, un bootstrap non paramétrique est effectué avec $B = 1\,000$ rééchantillons de taille $n = \#(\mathcal{S} \times \mathcal{T})$, où $\mathcal{S}$ désigne les emplacements (pays ou régions) et $\mathcal{T}$ les années (1990-2019). La statistique $g$ de Cohen, adaptée pour mesurer les disparités sexuelles, est calculée pour le $b$-ième rééchantillon comme suit :
 \begin{equation}
g_a^{(b)} = \frac{1}{n} \sum_{i=1}^n \mathrm{sign} \left( I_{s_i^{(b)}, t_i^{(b)}, a, \mathrm{mâle}}^{(b,i)} - I_{s_i^{(b)}, t_i^{(b)}, a, \mathrm{femelle}}^{(b,i)} \right),
 \end{equation}
où $I_{s_i^{(b)}, t_i^{(b)}, a, g}^{(b,i)} \sim \mathcal{N}(\mu_{s_i^{(b)}, t_i^{(b)}, a, g}, \sigma_{s_i^{(b)}, t_i^{(b)}, a, g}^2)$ représente la transformée logit (base 10) du taux d'incidence, et $\mathrm{sign}$ retourne $-1$, $0$ ou $+1$. La valeur $g_a$ est la moyenne des $g_a^{(b)}$, et sa distribution est approximée par $\mathcal{N}(\mu_a, \sigma_a^2)$, avec $\mu_a$ et $\sigma_a$ estimés à partir des $g_a^{(b)}$. Les intervalles de confiance à 95\,\% sont dérivés des quantiles empiriques (2,5\,\% et 97,5\,\%) des $g_a^{(b)}$, assurant une quantification robuste des disparités.

\subsection{Dynamique hyperbolique}

Les paires $(\mu_a, \sigma_a)$ sont projetées dans le demi-plan de Poincaré $\mathbb{H} = \{ z \in \mathbb{C} \mid \Im(z) > 0 \}$ via $z_a = \mu_a + i \sigma_a$. Cette représentation exploite la structure riemannienne de l'espace des distributions normales univariées, modélisant les variations par âge comme des trajectoires sur une variété hyperbolique, avec l'âge jouant le rôle de paramètre temporel.

\subsection{Dynamique symbolique}

Les variations qualitatives sont analysées via les signes des différences entre tranches d'âge consécutives (espacées de 5 ans) :
 \begin{equation}
S_a^{(\mu)} = \mathrm{sign}(\mu_a - \mu_{a-5}), \quad S_a^{(\sigma)} = \mathrm{sign}(\sigma_a - \sigma_{a-5}),
 \end{equation}
pour $a \geq 5$ (en tenant compte des tranches commençant à 0-4 ans). Ces indicateurs, calculés et agrégés par pays, mettent en évidence des motifs directionnels dans l'évolution des paramètres $\mu$ et $\sigma$.

\subsection{Distance de Fisher-Rao}

La distance de Fisher-Rao entre deux distributions $\mathcal{N}(\mu_1, \sigma_1^2)$ et $\mathcal{N}(\mu_2, \sigma_2^2)$ est donnée par l'équation~\eqref{eq:fr_normal} du chapitre~\ref{chap:math}. Les sommes cumulées de ces distances le long des trajectoires par âge quantifient l'ampleur globale des évolutions distributionnelles, une valeur plus élevée indiquant des changements plus marqués.

\subsection{Divergence de Kullback-Leibler}

La divergence de Kullback-Leibler, mesure asymétrique de dissimilarité, est définie par l'équation~\eqref{eq:kl_normal} du chapitre~\ref{chap:math}. Ses sommes cumulées le long des trajectoires évaluent la directionnalité des changements, une valeur plus élevée signalant une asymétrie accrue dans les transitions distributionnelles.

\section{L'effet de l'arsenic PM$_{2.5}$}
Cette section détaille les méthodes employées pour évaluer l'impact des concentrations d'arsenic dans les particules fines (PM$_{2.5}$) sur la prévalence des maladies respiratoires chroniques (MRC) aux États-Unis, en utilisant des données agrégées de GBD-2021 et de l'EPA. Les analyses reposent sur des agrégations temporelles, des régressions linéaires robustes et des projections géométriques dans l'espace hyperbolique, permettant de quantifier les associations stratifiées par âge et sexe.

\subsection{Agrégation de données}
Pour chaque combinaison de région $s \in \mathcal{S}$ (États des États-Unis), âge $a \in \mathcal{A}$ et sexe $g \in \mathcal{G}$, les paramètres de la prévalence transformée en logit (base 10) sont agrégés sur les années $t \in \mathcal{T}$ (1990-2019), avec $n = \#\mathcal{T}$ :
 \begin{equation}
\mu_{s,a,g} = \frac{1}{n} \sum_{k=1}^n \mu_{s,t_k,a,g}, \quad \sigma_{s,a,g} = \frac{1}{n} \sqrt{\sum_{k=1}^n \sigma_{s,t_k,a,g}^2},
 \end{equation}
où les éléments de $\mathcal{T}$ sont ordonnés chronologiquement comme $t_1 < t_2 < \dots < t_n$. Cette agrégation réduit la variabilité temporelle en combinant les moyennes et en ajustant les écarts-types pour refléter l'incertitude cumulée. Les concentrations d'arsenic $C_s \sim \mathcal{N}(\mu_s, \sigma_s^2)$ sont agrégées de manière similaire par État $s$, à partir des données de l'EPA.

\subsection{Méthode des moindres carrés ordinaires}
Pour chaque combinaison d'âge $a \in \mathcal{A}$ et de sexe $g \in \mathcal{G}$, les données agrégées de prévalence ($\mu_{s,a,g}$) et d'exposition à l'arsenic ($\mu_s$) sont préparées par filtrage et agrégation par région $s \in \mathcal{S}$, en excluant les valeurs manquantes, afin de former un ensemble de régression comportant au moins trois observations par combinaison $(a, g)$. Ces données sont renommées en $x = \mu_s$ et $y = \mu_{s,a,g}$.

Un modèle de régression linéaire ordinaire est ajusté par la méthode des moindres carrés : $y = \alpha + \beta x + \epsilon$, où $\epsilon \sim \mathcal{N}(0, \sigma^2)$. Les coefficients $\hat{\alpha}$ (ordonnée à l'origine) et $\hat{\beta}$ (pente) sont estimés, accompagnés de leurs valeurs $p$ associées. Les métriques de performance comprennent le coefficient de détermination $R^2$, le $R^2$ ajusté, la statistique $F$ et sa valeur $p$.

Des tests diagnostiques sont réalisés pour valider les hypothèses du modèle (voir sous-section des diagnostics du modèle dans le chapitre~\ref{chap:math}) :
\begin{itemize}
	\item Le test RESET évalue la spécification fonctionnelle (hypothèse nulle : forme linéaire adéquate).
	\item Le test de Shapiro-Wilk vérifie la normalité des résidus.
	\item Le test de Breusch-Pagan examine l'homoscédasticité des résidus.
	\item Le test de Durbin-Watson détecte l'autocorrélation des résidus.
	\item Le test des outliers, basé sur les résidus studentisés, identifie les valeurs aberrantes avec correction de Bonferroni.
\end{itemize}
Les points influents sont détectés via la distance de Cook (seuil : $4 / n$, où $n$ est le nombre d'observations) et le levier (seuil : $2p / n$, où $p = 2$ est le nombre de paramètres). Les mesures d'influence, incluant les DFFITS, sont calculées pour chaque observation.

Les résultats diagnostiques (coefficients, métriques, valeurs $p$ des tests, nombre d'outliers et de points influents) sont compilés dans un tableau récapitulatif par combinaison $(a, g)$. Les mesures d'influence détaillées par région sont également enregistrées. Ces sorties soutiennent des visualisations, telles que des histogrammes de $R^2$ et des valeurs $p$ de la pente, des graphiques de tendances par âge et sexe pour $R^2$, le RMSE (évalué par validation croisée à 10 plis, voir sous-section de la validation du modèle dans le chapitre~\ref{chap:math}), et le nombre de points influents, des boîtes à moustaches pour les leviers et distances de Cook, ainsi qu'une carte choroplèthe des leviers moyens par région. Des cartes de chaleur et des diagrammes de dispersion (pente vs. $-\log_{10}(p)$) complètent l'analyse afin d'identifier les associations significatives.

\subsection{Régression linéaire robuste}
Pour chaque paire $(a, g) \in \mathcal{A} \times \mathcal{G}$, deux modèles de régression linéaire robuste sont ajustés par la méthode des médianes répétées de Siegel : l'un reliant $\mu_{s,a,g}$ à $\mu_s$, et l'autre reliant $\log_{10} \sigma_{s,a,g}$ à $\mu_s$. Cette approche capture les effets sur la moyenne et la dispersion de la prévalence. Les estimateurs de pente $\hat{\beta}_{a,g}$ et d'ordonnée à l'origine $\hat{\alpha}_{a,g}$ sont obtenus via des médianes itérées des pentes et intercepts partiels, offrant une robustesse aux outliers avec un point de rupture asymptotique de 50\,\%.

Les intervalles de confiance à 95\,\% sont dérivés par bootstrap non paramétrique ($B = 1\,000$ rééchantillons des régions $s$), où les modèles sont réajustés à chaque itération. La déviation absolue médiane (MAD) des résidus, corrigée pour la consistance gaussienne, quantifie la dispersion robuste autour des ajustements. Des statistiques de test (valeur V et p-valeur associée) évaluent la significativité des coefficients $\hat{\beta}_{a,g}$ contre l'hypothèse nulle d'absence d'effet.

\subsection{Quantification géométrique des trajectoires}
Les estimateurs $\hat{\beta}_{a,g}$ et les MAD associées sont projetés dans le demi-plan de Poincaré $\mathbb{H} = \{ z \in \mathbb{C} \mid \Im(z) > 0 \}$ via $z_{a,g} = \hat{\beta}_{a,g} + i \cdot \mathrm{MAD}_{a,g}$, modélisant les évolutions par âge comme des trajectoires sur une variété hyperbolique. Les dissimilarités entre tranches d'âge consécutives sont mesurées par la distance de Fisher-Rao (équation~\eqref{eq:fr_normal} du chapitre~\ref{chap:math}) et la divergence de Kullback-Leibler (équation~\eqref{eq:kl_normal} du chapitre~\ref{chap:math}), avec leurs sommes cumulées quantifiant l'ampleur et la directionnalité des changements globaux, stratifiés par sexe.

Par ailleurs, les régions $s$ sont ordonnées par valeurs croissantes de $\mu_s$ et regroupées en clusters de tailles variables (par exemple, 1, 3 ou 17 régions par groupe). Pour chaque cluster, des paramètres agrégés de prévalence ($\mu$ et $\sigma$) sont calculés, formant des trajectoires dans le plan $(\mu, \sigma)$ qui visualisent les gradients d'effet liés à l'exposition croissante à l'arsenic, avec des barres d'erreur indiquant l'incertitude.
	\chapter{Résultats}
\label{chap:results}
\section{Trajectoires des paramètres et Fisher--Rao}
Insérez figures : trajectoires mu/sigma par âge et sexe, histogrammes de la somme cumulée des distances.


\section{Divergence de Kullback--Leibler}
Comparaison des tendances observées avec la métrique KL.


\section{Taille d'effet (g de Cohen) par tranche d'âge}
Graphiques et interprétation (prédominance masculine/enfance, inversion adolescence/jeunes adultes, variabilité >50 ans).


\section{Prévalence vs Incidence et exposition (PM2.5)}
Présentez le modèle linéaire sur la prévalence après transformation (log / logit), tableau des coefficients et diagnostics de résidus (Shapiro). Exemple d'insertion d'un tableau extrait du CSV :
%\begin{table}[H]
%	\centering
%	\caption{Extrait : résultats linéaires (arsenic\_linear.csv)}
%	\csvautotabular{arsenic_linear.csv}
%	\label{tab:arsenic_linear}
%\end{table}


\section{Analyses par pays et robustesse}
Comparer les signes (+1/-1) des changements par tranche d'âge et discuter de la robustesse.
	\chapter{Discussion}
\label{chap:discussion}

Ce chapitre discute et synthétise les résultats présentés au chapitre \ref{chap:results}, identifie les limites de l'étude et explore leurs implications théoriques et pratiques. Les analyses, fondées sur la géométrie de l'information, portent sur les données d'incidence et de prévalence des maladies respiratoires chroniques, stratifiées par âge, sexe et exposition régionale à l'arsenic contenu dans les PM$_{2,5}$.

\section{Synthèse des résultats}

Les résultats empiriques mettent en évidence des motifs cohérents des variations des MRC selon l'âge et le sexe, ainsi qu'un effet modérateur de l'exposition à l'arsenic. Dans l'analyse de l'effet de l'âge, stratifiée par sexe, les intervalles de confiance des taux d'incidence (Figure \ref{fig:CI-age-sex}) montrent une évolution en \emph{U} — une incidence plus élevée chez les plus jeunes et chez les plus âgés. Les trajectoires hyperboliques dans le demi-plan de Poincaré (Figures \ref{fig:age-sex_Poincare_US_UK} et \ref{fig:age-sex_Poincare_NO_IT}) révèlent une tendance au mouvement antihoraire.

Les signes des différences des paramètres $\mu$ et $\sigma$ (Tables \ref{tab:age-sex-symbol-mu-male}, \ref{tab:age-sex-symbol-mu-female}, \ref{tab:age-sex-symbol-sigma-male} et \ref{tab:age-sex-symbol-sigma-female}) suggèrent une constance du phénomène biologique entre pays.

Les sommes cumulées des distances de Fisher–Rao (Figure \ref{fig:age-sex_FR_cumulative}) et des divergences de Kullback–Leibler (Figure \ref{fig:age-sex_KL_cumulative}) sont systématiquement plus élevées chez les hommes que chez les femmes — une caractéristique observée de manière robuste à travers plusieurs pays.

Concernant l'effet du sexe, stratifié par âge, les intervalles de confiance de la statistique $g$ de Cohen (Figure \ref{fig:sex-age_plot_country-sex}) indiquent une prédominance masculine pendant l'enfance, une prédominance féminine à l'adolescence et chez les jeunes adultes, puis une variabilité inter-pays à partir de 50 ans, avec une tendance générale vers une prédominance masculine chez les sujets plus âgés. Les trajectoires hyperboliques associées confirment ces différences de dynamique entre sexes (Figure \ref{fig:sex-age_Poincare}). Les variations par tranche d'âge, codées par signes d'augmentation ou de diminution pour $\mu$ et $\sigma$, sont par ailleurs similaires entre pays (Tables \ref{tab:age-sex-symbol-mu} et \ref{tab:age-sex-symbol-sigma}).

Pour l'effet de l'arsenic dans les PM$*{2,5}$, les cartes et représentations géographiques (Figures \ref{fig:arsenic-map} et \ref{fig:arsenic_Poincare}) ainsi que les log-cotes de prévalence (Figures \ref{fig:GBD-map} et \ref{fig:GBD_Poincare}) soulignent des variations régionales marquées. Les intervalles de confiance des pentes $\hat{\beta}*{a,g}$ (Figure \ref{fig:CI-arsenic}) montrent une relation monotone entre l'exposition à l'arsenic présent dans les PM$_{2,5}$ et la prévalence des MRC ; cette relation dépend toutefois de l'âge et du sexe.

Les trajectoires dans le demi-plan de Poincaré (Figure \ref{fig:trajectoires-arsenic}), ainsi que les distances cumulées de Fisher–Rao (Figure \ref{fig:arsenic_distance_cumulative-Fisher-Rao}) et de Kullback–Leibler (Figure \ref{fig:arsenic_distance_cumulative-Kullback-Leibler}), montrent une variabilité plus prononcée chez les hommes.

\section{Limites de l'étude}

Des facteurs confondants potentiels — tabagisme, exposition à d'autres polluants atmosphériques (par exemple ozone, dioxyde d'azote) ou déterminants socio-économiques — n'ont pas été contrôlés et pourraient influencer les associations observées. 

L'utilisation de concentrations régionales d'arsenic dans les PM$_{2,5}$ comme proxy d'exposition présente des limites : ces mesures agrégées n'ont pas de résolution individuelle, et bien que des effets synergiques avec les PM$_{2,5}$ soient documentés, la voie d'exposition principale de l'arsenic reste souvent orale plutôt qu'inhalatoire.

De plus, les bases de données utilisées (GBD-2021 et EPA), couvrant la période 1990–2019, comportent des incertitudes de modélisation et une résolution spatiale limitée. L'approche par bootstrap employée pour estimer la variabilité suppose une indépendance qui pourrait être violée en présence de corrélations temporelles ou spatiales non modélisées. Enfin, cette décrit des corrélations et ne permet pas d'établir des relations causales.

\section{Implications théoriques}

Ces résultats s'inscrivent dans le cadre de la théorie évolutive de Geodakyan, qui postule une plus grande variabilité phénotypique chez les mâles en faveur d'une adaptation environnementale, tandis que les femelles conserveraient une stabilité phylogénétique. La variabilité accrue observée chez les hommes dans les trajectoires distributionnelles (Figures \ref{fig:age-sex_FR_cumulative} et \ref{fig:age-sex_KL_cumulative}), cohérente entre pays, apporte un soutien empirique quantitatif à cette perspective via la géométrie de l'information. Ces éléments contribuent au débat sur les fondements biologiques de la théorie et suggèrent d'éventuelles extensions à d'autres traits adaptatifs.

\section{Implications pratiques}

Sur le plan opérationnel, ces résultats ont des implications pour l'actuariat : les modèles de tarification des primes d'assurance santé pourraient intégrer les interactions âge–sexe–pollution afin d'affiner l'évaluation des risques au niveau régional. En épidémiologie, les méthodes fondées sur les distances de Fisher–Rao et les divergences de Kullback–Leibler peuvent être transposées à d'autres pathologies pour quantifier les différences sexuelles, et potentiellement informées a priori par la théorie de Geodakyan.

En sciences de l'environnement, la corrélation observée entre pollution atmosphérique et prévalence des MRC, associée à des indications selon lesquelles la sécheresse peut accroître la mobilisation de l'arsenic dans les sols et l'eau, souligne la nécessité d'étudier les impacts indirects du changement climatique sur la santé respiratoire. En santé publique et en politiques publiques, ces résultats plaident pour une allocation des ressources sanitaires mieux ciblée, fondée sur des données d'âge, de sexe et de niveaux régionaux de pollution plutôt que sur une distribution uniforme.

Ces implications motivent des recherches futures, notamment des études longitudinales et des approches permettant un meilleur contrôle des confondeurs, afin d'affiner les modèles et de confirmer les associations observées dans des contextes plus variés.

	\chapter{Conclusion et perspectives}
\label{chap:conclusion}
Résumé synthétique, recommandations pour actuariat / santé publique, pistes futures (autres polluants, analyses causales longitudinales).


%
	
	
	
	% -------------------------------------------------
	% Annexes
	% -------------------------------------------------
	%\appendix
	%%\chapter{Figures et tables complémentaires}

	
	\bigskip
	\bibliographystyle{plainnat}
	\bibliography{biblio}

\end{document}