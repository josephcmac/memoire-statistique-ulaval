\section{Abstract}

This thesis investigates the impact of regional concentrations of arsenic in fine particulate matter (PM${2.5}$) on the incidence and prevalence of chronic respiratory diseases (CRDs), accounting for variations by age and sex. Motivated by established links between arsenic exposure and respiratory conditions, as well as synergistic effects with PM${2.5}$, the study includes a preliminary analysis of age-sex interactions without considering arsenic, interpreted through Geodakyan’s evolutionary theory, which highlights greater variability in males as a mechanism for environmental adaptation.
The implications extend to actuarial science, for refining health insurance premium models, and to public health, for targeted resource allocation.
Empirically, the analyses reveal that the cumulative sum of Fisher-Rao distances between consecutive trajectories of incidence distribution parameters is consistently higher in men than in women, with similar patterns observed using Kullback-Leibler divergence; these findings are consistent across countries. The relationship between arsenic exposure and incidence is age-dependent, with a reversal in very elderly subjects due to diagnostic bias, while prevalence increases linearly after appropriate transformations. Methodologically, the approach relies on information geometry (Fisher-Rao distance, Kullback-Leibler divergence), effect size via Cohen’s g for proportions, and bootstrapping to estimate variability.
Interpreted within Geodakyan’s theoretical framework, these results contribute to the debate on its biological foundations and open practical perspectives in actuarial science (premium adjustment), epidemiology (extension to other diseases), environmental science (links to climate change increasing airborne arsenic concentrations via drought), and political science (optimized distribution of health resources). The thesis is structured into chapters covering mathematical preliminaries, data, methodology, results, discussion, conclusions, and methodological appendices.