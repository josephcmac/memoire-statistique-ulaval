\section{Abstract}

This master's dissertation investigates sex-related differences in the trajectories of statistical parameters and the relationship between atmospheric exposure (PM2.5 — proxy for arsenic) and chronic respiratory diseases, combining information geometry methods, effect size indices, and epidemiological analyses on multi-country datasets.

\paragraph{Methods.} To quantify temporal variations of parameters $(\mu,\sigma)$, I use the Fisher–Rao distance applied to parametric families, complemented by the Kullback–Leibler divergence as an alternative measure. Effect sizes (Cohen’s g) are computed by age group; exposure–prevalence relationships are modeled via linear regressions on log- and logit-transformed prevalence. Diagnostics include Shapiro–Wilk tests for residual normality and corrections for multiple comparisons.

\paragraph{Main results.} The cumulative sum of successive Fisher–Rao distances is consistently higher for men than for women, a property confirmed across several countries; analogous findings emerge with the KL divergence. Analysis of effect sizes (Cohen’s g) shows male predominance during childhood, female predominance during adolescence and young adulthood, followed by inter-country variability after age 50, with a general trend toward male predominance. For arsenic exposure (PM2.5), incidence varies with age: around 47 years (±2), incidence increases with arsenic concentration among men, while in very old men (\~92 ±2 years) the relationship reverses — a bias explained by accumulation and earlier diagnosis. In contrast, prevalence increases robustly with pollution; the prevalence–pollution relationship appears linear after log / logit transformations. Residual normality diagnostics are highly satisfactory, subject to multiple-testing corrections.

\paragraph{Interpretation and implications.} I interpret these empirical patterns in light of Geodakyan’s biological theory, which offers a framework to explain observed sex differences. The results have practical implications for actuarial science (health insurance pricing by age/sex/pollution), epidemiology (transferable methods to other diseases), and environmental health policy (targeted resource allocation). Data and R scripts are provided in the appendix. Future work could extend the study to other pollutants and implement causal longitudinal approaches.
