\section*{Avant-propos}
Après avoir déménagé à la ville de Québec pour poursuivre mes études à l’Université Laval, j’ai observé que, particulièrement durant l’été, la qualité de l’air était vraiment mauvaise et dangereuse pour la santé. Cette constatation m’a amené à me poser la question suivante : l’effet de la pollution atmosphérique sur la santé est-il négligeable ou non ? Ce mémoire offre une réponse partielle à cette interrogation générale. J’espère que ma contribution favorisera une prise de conscience écologique accrue et stimulera une volonté politique visant à réduire la pollution atmosphérique. Dans les situations où celle-ci s’avère inévitable, telles que les incendies naturels, il convient d’adopter des mesures pour limiter l’exposition humaine, par exemple en recourant à des masques et à des purificateurs d’air. De plus, compte tenu de l’effet du changement climatique sur la concentration atmosphérique d’arsenic à cause de la sécheresse, j’espère que cette thèse motivera la recherche sur les effets directs et indirects du changement climatique sur la santé humaine.