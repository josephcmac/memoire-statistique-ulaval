\chapter{Méthodes}
\label{chap:methods}

\section{Usage de l'intelligence artificielle générative}
\label{sec:ia}

Des outils d'intelligence artificielle générative (p.ex. ChatGPT-o1 Pro, Grok4-heavy) ont été sollicités ponctuellement pour la reformulation de textes, la génération de brouillons, l'aide au prototypage de code et la recherche documentaire. Toutes les analyses, les choix méthodologiques et les interprétations présentés dans ce mémoire ont été réalisés et validés par l'auteur.

Les principaux \emph{prompts} utilisés sont les suivants :

\begin{enumerate}
	\item \textbf{Rédaction en \LaTeX{}} :
	\begin{quote}
		« Vous êtes un statisticien professionnel. Sans changer les idées, améliorez l'orthographe, la grammaire et le style du texte suivant en \LaTeX{} (français natif). »
	\end{quote}
	\item \textbf{Programmation en R} :
	\begin{quote}
		« Vous êtes un statisticien professionnel. Comment transformer les données décrites ci-dessous en R en suivant les bonnes pratiques ? »
	\end{quote}
	\item \textbf{Recherche documentaire / moteur de recherche} :
	\begin{quote}
		« Vous êtes un statisticien professionnel. Proposez des références et un bref résumé des travaux scientifiques pertinents sur [sujet]. »
	\end{quote}
	\item \textbf{Vérification et correction} :
	\begin{quote}
		« Vous êtes un statisticien professionnel. Y a-t-il des erreurs factuelles, méthodologiques ou syntaxiques dans le passage ci-dessous ? »
	\end{quote}
		\item \textbf{Transformation de format} :
	\begin{quote}
		« Convertissez les références ci-dessous au format BibTeX. »
	\end{quote}
\end{enumerate}

Les suggestions bibliographiques ou synthèses proposées par l'intelligence artificielle générative ont fait l'objet d'une vérification systématique dans les sources primaires et les bases de données académiques avant d'être retenues. Les \emph{prompts} non triviaux ainsi que les sorties pertinentes ont été archivés dans le dépôt de travail associé au mémoire. Il n'est pas jugé pertinent de conserver les \emph{prompts} triviaux\footnote{L'un des objectifs de l'intelligence artificielle générative est de réduire la charge de travail académique. Accorder trop d'importance aux utilisations triviales de cette technologie est contraire à son objectif. Bien entendu, les applications non triviales doivent être documentées.}, tels que les corrections grammaticales, les recherches documentaires simples, qui peuvent être effectuées via Google ou la correction des erreurs en R. Seuls les \emph{prompts} liés à l'interprétation des données ou à des propositions méthodologiques ont été conservés. Aucune donnée individuelle identifiable ni information confidentielle n’a été transmise à des services externes sans garanties appropriées.


