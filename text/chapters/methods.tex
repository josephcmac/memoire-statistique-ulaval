\chapter{Méthodes}
\label{chap:methods}
Ce chapitre détaille les approches méthodologiques employées pour analyser les données sur les maladies respiratoires chroniques (MRC), issues de la base GBD-2021. Les méthodes sont organisées par effet étudié : l'âge stratifié par sexe, le sexe stratifié par âge, et l'exposition à l'arsenic dans les PM$_{2.5}$. Elles reposent sur un cadre probabiliste où les taux transformés en logit (base 10) sont modélisés par des distributions normales, facilitant l'application de techniques géométriques et statistiques robustes.

\section{L'effet de l'âge stratifié par le sexe}

Cette section détaille les méthodes employées pour évaluer l'impact de l'âge sur les taux d'incidence des maladies respiratoires chroniques (MRC), stratifiés par sexe. Les analyses reposent sur des agrégations bootstrap des données GBD-2021, des projections dans l'espace hyperbolique et des mesures de dissimilarité géométrique, appliquées par pays lorsque pertinent.

\subsection{Agrégation de données}

Pour chaque paire $(a, g) \in \mathcal{A} \times \mathcal{G}$, avec $\mathcal{A}$ l'ensemble des tranches d'âge et $\mathcal{G} = \{\text{mâle}, \text{femelle}\}$, un bootstrap non paramétrique est effectué avec $B = 1\,000$ rééchantillons de taille $n = \#(\mathcal{S} \times \mathcal{T})$, où $\mathcal{S}$ désigne les emplacements (pays ou régions) et $\mathcal{T}$ les années (1990-2019). Pour le $b$-ième rééchantillon, la moyenne est calculée comme suit :
\[
I_{a,g}^{(b)} = \frac{1}{n} \sum_{i=1}^n I_{s_i^{(b)}, t_i^{(b)}, a, g}^{(b,i)},
\]
où $I_{s_i^{(b)}, t_i^{(b)}, a, g}^{(b,i)} \sim \mathcal{N}(\mu_{s_i^{(b)}, t_i^{(b)}, a, g}, \sigma_{s_i^{(b)}, t_i^{(b)}, a, g}^2)$ représente la transformée logit (base 10) du taux d'incidence. Les paramètres $\mu_{a,g}$ et $\sigma_{a,g}$ sont estimés comme la moyenne et l'écart-type des $I_{a,g}^{(b)}$. Le taux d'incidence moyen $i_{a,g}$ est obtenu par inversion :
\[
i_{a,g} = \frac{10^5}{1 + 10^{-\mu_{a,g}}},
\]
avec des intervalles de confiance à 95\,\% dérivés des quantiles empiriques (2,5\,\% et 97,5\,\%) des valeurs transformées, assurant une quantification robuste de l'incertitude sur l'échelle des taux par 100\,000 habitants.

\subsection{Dynamique hyperbolique}

Les paires $(\mu_{a,g}, \sigma_{a,g})$ sont projetées dans le demi-plan de Poincaré $\mathbb{H} = \{ z \in \mathbb{C} \mid \Im(z) > 0 \}$ via $z_{a,g} = \mu_{a,g} + i \sigma_{a,g}$. Cette représentation exploite la structure riemannienne de l'espace des distributions normales univariées, modélisant les variations par âge comme des trajectoires sur une variété hyperbolique, avec l'âge jouant le rôle de paramètre temporel.

\subsection{Dynamique symbolique}

Les variations qualitatives sont analysées via les signes des différences entre tranches d'âge consécutives (espacées de 5 ans) :
\[
S_{a,g}^{(\mu)} = \mathrm{sign} (\mu_{a,g} - \mu_{a-5,g}), \quad S_{a,g}^{(\sigma)} = \mathrm{sign}(\sigma_{a,g} - \sigma_{a-5,g}),
\]
pour $a \geq 5$ (en tenant compte des tranches commençant à 0-4 ans). Ces indicateurs, calculés et agrégés par pays, mettent en évidence des motifs directionnels dans l'évolution des paramètres $\mu$ et $\sigma$.

\subsection{Distance de Fisher-Rao}

La distance de Fisher-Rao entre deux distributions $\mathcal{N}(\mu_1, \sigma_1^2)$ et $\mathcal{N}(\mu_2, \sigma_2^2)$ est donnée par l'équation~\eqref{eq:fr_normal} du chapitre~\ref{chap:math}. Les sommes cumulées de ces distances le long des trajectoires par âge et sexe quantifient l'ampleur globale des évolutions distributionnelles, une valeur plus élevée indiquant des changements plus marqués.

\subsection{Divergence de Kullback-Leibler}

La divergence de Kullback-Leibler, mesure asymétrique de dissimilarité, est définie par l'équation~\eqref{eq:kl_normal} du chapitre~\ref{chap:math}. Ses sommes cumulées le long des trajectoires évaluent la directionnalité des changements, une valeur plus élevée signalant une asymétrie accrue dans les transitions distributionnelles.

\section{L'effet du sexe stratifié par l'âge}

Cette section détaille les méthodes employées pour évaluer l'impact du sexe sur les taux d'incidence des maladies respiratoires chroniques (MRC), stratifiés par âge. Les analyses reposent sur des agrégations bootstrap des données GBD-2021, des projections dans l'espace hyperbolique et des mesures de dissimilarité géométrique, appliquées par pays lorsque pertinent.

\subsection{Agrégation de données}

Pour chaque âge $a \in \mathcal{A}$, un bootstrap non paramétrique est effectué avec $B = 1\,000$ rééchantillons de taille $n = \#(\mathcal{S} \times \mathcal{T})$, où $\mathcal{S}$ désigne les emplacements (pays ou régions) et $\mathcal{T}$ les années (1990-2019). La statistique $g$ de Cohen, adaptée pour mesurer les disparités sexuelles, est calculée pour le $b$-ième rééchantillon comme suit :
\[
g_a^{(b)} = \frac{1}{n} \sum_{i=1}^n \mathrm{sign} \left( I_{s_i^{(b)}, t_i^{(b)}, a, \mathrm{mâle}}^{(b,i)} - I_{s_i^{(b)}, t_i^{(b)}, a, \mathrm{femelle}}^{(b,i)} \right),
\]
où $I_{s_i^{(b)}, t_i^{(b)}, a, g}^{(b,i)} \sim \mathcal{N}(\mu_{s_i^{(b)}, t_i^{(b)}, a, g}, \sigma_{s_i^{(b)}, t_i^{(b)}, a, g}^2)$ représente la transformée logit (base 10) du taux d'incidence, et $\mathrm{sign}$ retourne $-1$, $0$ ou $+1$. La valeur $g_a$ est la moyenne des $g_a^{(b)}$, et sa distribution est approximée par $\mathcal{N}(\mu_a, \sigma_a^2)$, avec $\mu_a$ et $\sigma_a$ estimés à partir des $g_a^{(b)}$. Les intervalles de confiance à 95\,\% sont dérivés des quantiles empiriques (2,5\,\% et 97,5\,\%) des $g_a^{(b)}$, assurant une quantification robuste des disparités.

\subsection{Dynamique hyperbolique}

Les paires $(\mu_a, \sigma_a)$ sont projetées dans le demi-plan de Poincaré $\mathbb{H} = \{ z \in \mathbb{C} \mid \Im(z) > 0 \}$ via $z_a = \mu_a + i \sigma_a$. Cette représentation exploite la structure riemannienne de l'espace des distributions normales univariées, modélisant les variations par âge comme des trajectoires sur une variété hyperbolique, avec l'âge jouant le rôle de paramètre temporel.

\subsection{Dynamique symbolique}

Les variations qualitatives sont analysées via les signes des différences entre tranches d'âge consécutives (espacées de 5 ans) :
\[
S_a^{(\mu)} = \mathrm{sign}(\mu_a - \mu_{a-5}), \quad S_a^{(\sigma)} = \mathrm{sign}(\sigma_a - \sigma_{a-5}),
\]
pour $a \geq 5$ (en tenant compte des tranches commençant à 0-4 ans). Ces indicateurs, calculés et agrégés par pays, mettent en évidence des motifs directionnels dans l'évolution des paramètres $\mu$ et $\sigma$.

\subsection{Distance de Fisher-Rao}

La distance de Fisher-Rao entre deux distributions $\mathcal{N}(\mu_1, \sigma_1^2)$ et $\mathcal{N}(\mu_2, \sigma_2^2)$ est donnée par l'équation~\eqref{eq:fr_normal} du chapitre~\ref{chap:math}. Les sommes cumulées de ces distances le long des trajectoires par âge quantifient l'ampleur globale des évolutions distributionnelles, une valeur plus élevée indiquant des changements plus marqués.

\subsection{Divergence de Kullback-Leibler}

La divergence de Kullback-Leibler, mesure asymétrique de dissimilarité, est définie par l'équation~\eqref{eq:kl_normal} du chapitre~\ref{chap:math}. Ses sommes cumulées le long des trajectoires évaluent la directionnalité des changements, une valeur plus élevée signalant une asymétrie accrue dans les transitions distributionnelles.

\section{L'effet de l'arsenic PM$_{2.5}$}

Cette section détaille les méthodes employées pour évaluer l'impact des concentrations d'arsenic dans les particules fines (PM$_{2.5}$) sur la prévalence des maladies respiratoires chroniques (MRC) aux États-Unis, en utilisant des données agrégées de GBD-2021 et de l'EPA. Les analyses reposent sur des agrégations temporelles, des régressions linéaires robustes et des projections géométriques dans l'espace hyperbolique, permettant de quantifier les associations stratifiées par âge et sexe.

\subsection{Agrégation de données}

Pour chaque combinaison de région $s \in \mathcal{S}$ (États des États-Unis), âge $a \in \mathcal{A}$ et sexe $g \in \mathcal{G}$, les paramètres de la prévalence transformée en logit (base 10) sont agrégés sur les années $t \in \mathcal{T}$ (1990-2019), avec $n = \#\mathcal{T}$ :
\[
\mu_{s,a,g} = \frac{1}{n} \sum_{k=1}^n \mu_{s,t_k,a,g}, \quad \sigma_{s,a,g} = \frac{1}{n} \sqrt{\sum_{k=1}^n \sigma_{s,t_k,a,g}^2},
\]
où les éléments de $\mathcal{T}$ sont ordonnés chronologiquement comme $t_1 < t_2 < \dots < t_n$. Cette agrégation réduit la variabilité temporelle en combinant les moyennes et en ajustant les écarts-types pour refléter l'incertitude cumulée. Les concentrations d'arsenic $C_s \sim \mathcal{N}(\mu_s, \sigma_s^2)$ sont agrégées de manière similaire par État $s$, à partir des données de l'EPA.

\subsection{Régression linéaire robuste}

Pour chaque paire $(a, g) \in \mathcal{A} \times \mathcal{G}$, deux modèles de régression linéaire robuste sont ajustés par la méthode des médianes répétées de Siegel : l'un reliant $\mu_{s,a,g}$ à $\mu_s$, et l'autre reliant $\log_{10} \sigma_{s,a,g}$ à $\mu_s$. Cette approche capture les effets sur la moyenne et la dispersion de la prévalence. Les estimateurs de pente $\hat{\beta}_{a,g}$ et d'ordonnée à l'origine $\hat{\alpha}_{a,g}$ sont obtenus via des médianes itérées des pentes et intercepts partiels, offrant une robustesse aux outliers avec un point de rupture asymptotique de 50\,\%. 

Les intervalles de confiance à 95\,\% sont dérivés par bootstrap non paramétrique ($B = 1\,000$ rééchantillons des régions $s$), où les modèles sont réajustés à chaque itération. La déviation absolue médiane (MAD) des résidus, corrigée pour la consistance gaussienne, quantifie la dispersion robuste autour des ajustements. Des statistiques de test (valeur V et p-valeur associée) évaluent la significativité des coefficients $\hat{\beta}_{a,g}$ contre l'hypothèse nulle d'absence d'effet.

\subsection{Quantification géométrique des trajectoires}

Les estimateurs $\hat{\beta}_{a,g}$ et les MAD associées sont projetés dans le demi-plan de Poincaré $\mathbb{H} = \{ z \in \mathbb{C} \mid \Im(z) > 0 \}$ via $z_{a,g} = \hat{\beta}_{a,g} + i \cdot \mathrm{MAD}_{a,g}$, modélisant les évolutions par âge comme des trajectoires sur une variété hyperbolique. Les dissimilarités entre tranches d'âge consécutives sont mesurées par la distance de Fisher-Rao (équation~\eqref{eq:fr_normal} du chapitre~\ref{chap:math}) et la divergence de Kullback-Leibler (équation~\eqref{eq:kl_normal} du chapitre~\ref{chap:math}), avec leurs sommes cumulées quantifiant l'ampleur et la directionnalité des changements globaux, stratifiés par sexe.

Par ailleurs, les régions $s$ sont ordonnées par valeurs croissantes de $\mu_s$ et regroupées en clusters de tailles variables (par exemple, 1, 3 ou 17 régions par groupe). Pour chaque cluster, des paramètres agrégés de prévalence ($\mu$ et $\sigma$) sont calculés, formant des trajectoires dans le plan $(\mu, \sigma)$ qui visualisent les gradients d'effet liés à l'exposition croissante à l'arsenic, avec des barres d'erreur indiquant l'incertitude.