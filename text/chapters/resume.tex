\section{Résumé}

Ce mémoire examine l'impact de la concentration régionale d'arsenic dans les particules fines PM$_{2{,}5}$ sur l'incidence et la prévalence des maladies respiratoires chroniques (MRC), en tenant compte des variations selon l'âge et le sexe. Motivé par les liens établis entre l'ingestion d'arsenic et les affections respiratoires, ainsi que par les effets synergiques avec les PM$_{2{,}5}$, l'étude intègre une analyse préliminaire des interactions âge-sexe sans considération de l'arsenic, interprétées à travers la théorie évolutive de Geodakyan, qui souligne une plus grande variabilité chez les mâles pour favoriser l'adaptation environnementale. Les implications s'étendent à l'actuariat, pour affiner les modèles de primes d'assurance santé, et à la santé publique, pour une allocation ciblée des ressources.

Sur le plan empirique, les analyses révèlent que la somme cumulée des distances de Fisher-Rao entre trajectoires consécutives des paramètres de distribution des incidences est systématiquement plus élevée chez les hommes que chez les femmes, avec des résultats similaires via la divergence de Kullback-Leibler ; ces motifs sont consistants entre pays. La relation entre exposition à l'arsenic et incidence dépend de l'âge, avec une inversion chez les sujets très âgés due à un biais de diagnostic, tandis que la prévalence augmente linéairement après transformations. Méthodologiquement, l'approche repose sur la géométrie de l'information (distance de Fisher-Rao, divergence de Kullback-Leibler), la taille d'effet via le \( g \) de Cohen pour les proportions, et le bootstrapping pour estimer la variabilité.

Interprétés dans le cadre de la théorie de Geodakyan, ces résultats contribuent au débat sur ses fondements biologiques et ouvrent des perspectives pratiques en actuariat (ajustement des primes), épidémiologie (extension à d'autres pathologies), environnement (liens avec le changement climatique augmentant les concentrations d'arsenic aérien via la sécheresse) et sciences politiques (distribution optimisée des ressources sanitaires). Le mémoire est structuré en chapitres couvrant les préliminaires mathématiques, les données, la méthodologie, les résultats, la discussion, les conclusions et des annexes méthodologiques.