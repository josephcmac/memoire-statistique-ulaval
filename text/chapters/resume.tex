\section{Résumé}

Ce mémoire étudie les différences liées au sexe dans les trajectoires de paramètres statistiques et la relation entre exposition atmosphérique (PM2.5 — proxy arsenic) et maladies respiratoires chroniques, en combinant méthodes géométriques de l’information, indices d’effet et analyses épidémiologiques sur jeux de données multi-pays.

\paragraph{Méthodes.} Pour quantifier les variations temporelles des paramètres $(\mu,\sigma)$ j’utilise la distance de Fisher–Rao appliquée à des familles paramétriques, complétée par la divergence de Kullback–Leibler comme mesure alternative. Les tailles d’effet (g de Cohen) sont calculées par tranche d’âge ; les relations exposition–prévalence sont modélisées par régressions linéaires appliquées à des transformations log et logit de la prévalence. Les diagnostics incluent les tests de normalité de Shapiro–Wilk sur les résidus et des ajustements pour comparaisons multiples.

\paragraph{Résultats principaux.} La somme cumulée des distances consécutives Fisher–Rao est systématiquement plus élevée chez les hommes que chez les femmes, propriété confirmée sur plusieurs pays ; observation analogue obtenue avec la divergence KL. L’analyse des tailles d’effet (g de Cohen) montre une prédominance masculine durant l’enfance, une prédominance féminine à l’adolescence et chez les jeunes adultes, puis une variabilité inter-pays après 50 ans avec une tendance générale vers une prédominance masculine. Pour l’exposition à l’arsenic (PM2.5), l’incidence varie avec l’âge : autour de 47 ans (±2) l’incidence augmente avec la concentration d’arsenic chez les hommes, tandis que chez des hommes très âgés (\~92 ±2 ans) la relation s’inverse — un biais expliqué par un effet d’accumulation et de diagnostic antérieur. En revanche, la prévalence augmente de manière robuste avec la pollution ; la relation prévalence–pollution apparaît linéaire après transformations log / logit. Les diagnostics de normalité des résidus sont très satisfaisants, sous réserve d’une correction pour tests multiples.

\paragraph{Interprétation et implications.} J’interprète ces motifs empiriques à la lumière de la théorie biologique de Geodakyan, qui fournit un cadre pour expliquer les différences sexuelles observées. Les résultats ont des implications pratiques pour l’actuariat (tarification liée à âge/sex/pollution), l’épidémiologie (méthodes transférables à d’autres pathologies) et les politiques de santé environnementale (allocation ciblée des ressources). Les données et les scripts R sont fournis en annexe. Des travaux futurs pourraient étendre l’étude à d’autres polluants et mettre en œuvre approches causales longitudinales.
