\chapter{Discussion}
\label{chap:discussion}

Ce chapitre discute et synthétise les résultats présentés au chapitre \ref{chap:results}, identifie les limites de l'étude et explore leurs implications théoriques et pratiques. Les analyses, fondées sur la géométrie de l'information, portent sur les données d'incidence et de prévalence des maladies respiratoires chroniques, stratifiées par âge, sexe et exposition régionale à l'arsenic contenu dans les PM$_{2,5}$.

\section{Synthèse des résultats}

Les résultats empiriques révèlent des motifs cohérents dans les variations des maladies respiratoires chroniques en fonction de l'âge, du sexe et de l'exposition à l'arsenic PM$_{2,5}$. Dans l'analyse de l'effet de l'âge stratifié par le sexe, les intervalles de confiance des taux d'incidence (Figure \ref{fig:CI-age-sex}) indiquent une évolution en forme de U, avec des taux plus élevés chez les très jeunes et les personnes âgées, plus marquée chez les mâles que chez les femelles. Les trajectoires hyperboliques dans le demi-plan de Poincaré (Figures \ref{fig:age-sex_Poincare_US_UK} et \ref{fig:age-sex_Poincare_NO_IT}) exhibent des mouvements généralement antihoraire, reflétant des changements dans les paramètres ($\mu, \sigma$) au fil de l'âge, avec des motifs similaires à travers les pays étudiés (États-Unis, Royaume-Uni, Norvège, Italie).

Les signes des différences entre tranches d'âge consécutives pour $\mu$ et $\sigma$ (Tables \ref{tab:age-sex-symbol-mu-male}, \ref{tab:age-sex-symbol-mu-female}, \ref{tab:age-sex-symbol-sigma-male} et \ref{tab:age-sex-symbol-sigma-female}) démontrent une constance qui semble être de nature biologique plutôt qu’environnementale. Les sommes cumulées des distances de Fisher-Rao (Figure \ref{fig:age-sex_FR_cumulative}) et des divergences de Kullback-Leibler (Figure \ref{fig:age-sex_KL_cumulative}) sont systématiquement plus élevées chez les hommes, soulignant une plus grande variabilité distributionnelle masculine, observée de manière robuste à travers les pays.

Pour l'effet du sexe stratifié par l'âge, les intervalles de confiance de la statistique g de Cohen (Figure \ref{fig:sex-age_plot_country-sex}) mettent en évidence une prédominance masculine en enfance, féminine à l'adolescence et chez les jeunes adultes, puis une variabilité inter-pays à partir de 50 ans, avec une tendance à la prédominance masculine chez les âgés. Les trajectoires hyperboliques associées (Figure \ref{fig:sex-age_Poincare}) semblent être un U inversé qui oscille d'un côté à l'autre de la ligne $\mu = 0$, rappelant la trajectoire du tige de pendule d'un métronome mécanique. Les signes des changements pour $\mu$ et $\sigma$ (Tables \ref{tab:age-sex-symbol-mu} et \ref{tab:age-sex-symbol-sigma}) sont similaires entre pays, renforçant la généralisabilité.

Concernant l'effet de l'arsenic PM$_{2,5}$, les distributions géographiques (Figures \ref{fig:arsenic-map} et \ref{fig:arsenic_Poincare}) montrent des concentrations moyennes variant par État aux États-Unis, avec des paramètres ($\mu_s, \sigma_s$) indiquant des hétérogénéités régionales. Les log-cotes moyennes de prévalence des MRC (Figures \ref{fig:GBD-map} et \ref{fig:GBD_Poincare}) révèlent des variations spatiales marquées, avec des paramètres ($\mu_{s,a,g}, \sigma_{s,a,g}$) soulignant des disparités. 

Les analyses de validation et de diagnostic, centrées sur les modèles de régression linéaire des moindres carrés ordinaires pour l'effet de l'arsenic, confirment la robustesse globale des résultats. Les métriques de performance, telles que la distribution du coefficient de détermination $R^2$ (Figure~\ref{fig:global_r2}) et les tendances par âge et sexe (Figure~\ref{fig:trends_r2}), indiquent un ajustement satisfaisant, avec une proportion élevée d'associations significatives (Figure~\ref{fig:global_p}). Les tests diagnostiques, incluant le RESET pour la linéarité (Figure~\ref{fig:reset_hist}), le Shapiro-Wilk pour la normalité (Figure~\ref{fig:shapiro_hist}), le Breusch-Pagan pour l'homoscédasticité (Figure~\ref{fig:bp_hist}) et le Durbin-Watson pour l'autocorrélation (Figure~\ref{fig:dw_hist}), valident les hypothèses sous-jacentes pour la majorité des modèles, bien que des valeurs aberrantes et points influents dans certains sous-groupes (Figures~\ref{fig:outliers}, \ref{fig:leverage} et \ref{fig:cooks}) suggèrent une sensibilité limitée. La régression robuste complémentaire (méthode de Siegel), commentée ci-dessous, renforce ces conclusions, assurant la fiabilité des associations observées.

 En effet, les intervalles de confiance des pentes $\hat{\beta}_{a,g}$ (Figure \ref{fig:CI-arsenic}) indiquent une relation positive monotone avec l'exposition, modulée par l'âge et le sexe. Les trajectoires dans le demi-plan de Poincaré (Figure \ref{fig:trajectoires-arsenic}) et les distances cumulées de Fisher-Rao (Figure \ref{fig:arsenic_distance_cumulative-Fisher-Rao}) et Kullback-Leibler (Figure \ref{fig:arsenic_distance_cumulative-Kullback-Leibler}) exhibent une variabilité accrue chez les hommes, suggérant une sensibilité sexuelle différentiée à l'exposition.

Par conséquent, la régression linéaire (ordinaire ou robuste) fournit un bon modèle explicatif de l’effet de l’exposition à l’arsenic PM$_{2.5}$ sur la prévalence des maladies respiratoires chroniques, mais ce modèle trop simple est faible pour faire des prédictions.

\section{Limites de l'étude}

Des facteurs confondants potentiels — tabagisme, exposition à d'autres polluants atmosphériques (par exemple ozone, dioxyde d'azote) ou déterminants socio-économiques — n'ont pas été contrôlés et pourraient influencer les associations observées. 

L'utilisation de concentrations régionales d'arsenic dans les PM$_{2,5}$ comme proxy d'exposition présente des limites : ces mesures agrégées n'ont pas de résolution individuelle, et bien que des effets synergiques avec les PM$_{2,5}$ soient documentés, la voie d'exposition principale de l'arsenic reste souvent orale plutôt qu'inhalatoire.

De plus, les bases de données utilisées (GBD-2021 et EPA), couvrant la période 1990–2019, comportent des incertitudes de modélisation et une résolution spatiale limitée. L'approche par bootstrap employée pour estimer la variabilité suppose une indépendance qui pourrait être violée en présence de corrélations temporelles ou spatiales non modélisées. Enfin, cette décrit des corrélations et ne permet pas d'établir des relations causales.

\section{Implications théoriques}

Ces résultats s'inscrivent dans le cadre de la théorie évolutive de Geodakyan, qui postule une plus grande variabilité phénotypique chez les mâles en faveur d'une adaptation environnementale, tandis que les femelles conserveraient une stabilité phylogénétique. La variabilité accrue observée chez les hommes dans les trajectoires distributionnelles (Figures \ref{fig:age-sex_FR_cumulative} et \ref{fig:age-sex_KL_cumulative}), cohérente entre pays, apporte un soutien empirique quantitatif à cette perspective via la géométrie de l'information. Ces éléments contribuent au débat sur les fondements biologiques de la théorie et suggèrent d'éventuelles extensions à d'autres traits adaptatifs.

\section{Implications pratiques}

Sur le plan opérationnel, ces résultats ont des implications pour l'actuariat : les modèles de tarification des primes d'assurance santé pourraient intégrer les interactions âge–sexe–pollution afin d'affiner l'évaluation des risques au niveau régional. En épidémiologie, les méthodes fondées sur les distances de Fisher–Rao et les divergences de Kullback–Leibler peuvent être transposées à d'autres pathologies pour quantifier les différences sexuelles, et potentiellement informées a priori par la théorie de Geodakyan.

En sciences de l'environnement, la corrélation observée entre pollution atmosphérique et prévalence des MRC, associée à des indications selon lesquelles la sécheresse peut accroître la mobilisation de l'arsenic dans les sols et l'eau, souligne la nécessité d'étudier les impacts indirects du changement climatique sur la santé respiratoire. En santé publique et en politiques publiques, ces résultats plaident pour une allocation des ressources sanitaires mieux ciblée, fondée sur des données d'âge, de sexe et de niveaux régionaux de pollution plutôt que sur une distribution uniforme.

Ces implications motivent des recherches futures, notamment des études longitudinales et des approches permettant un meilleur contrôle des confondeurs, afin d'affiner les modèles et de confirmer les associations observées dans des contextes plus variés.
