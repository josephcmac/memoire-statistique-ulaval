\chapter{Résultats}
\label{chap:results}
\section{Trajectoires des paramètres et Fisher--Rao}
Insérez figures : trajectoires mu/sigma par âge et sexe, histogrammes de la somme cumulée des distances.


\section{Divergence de Kullback--Leibler}
Comparaison des tendances observées avec la métrique KL.


\section{Taille d'effet (g de Cohen) par tranche d'âge}
Graphiques et interprétation (prédominance masculine/enfance, inversion adolescence/jeunes adultes, variabilité >50 ans).


\section{Prévalence vs Incidence et exposition (PM2.5)}
Présentez le modèle linéaire sur la prévalence après transformation (log / logit), tableau des coefficients et diagnostics de résidus (Shapiro). Exemple d'insertion d'un tableau extrait du CSV :
%\begin{table}[H]
%	\centering
%	\caption{Extrait : résultats linéaires (arsenic\_linear.csv)}
%	\csvautotabular{arsenic_linear.csv}
%	\label{tab:arsenic_linear}
%\end{table}


\section{Analyses par pays et robustesse}
Comparer les signes (+1/-1) des changements par tranche d'âge et discuter de la robustesse.