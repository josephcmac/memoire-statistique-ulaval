\chapter{Résultats}
\label{chap:results}

Ce chapitre présente les résultats empiriques obtenus à partir des analyses décrites dans le chapitre \ref{chap:methods}. Les résultats sont organisés en sections correspondant aux approches méthodologiques principales.

\section{L'effet de l'âge stratifié par le sexe}

\subsection{Intervales de confiance}

Les intervales de confiance des taux d'incidence des MRC sont présentés dans les Figures \ref{fig:CI-age-sex-males} et \ref{fig:CI-age-sex-females}.

\begin{figure}[H]
	\centering
	\includegraphics[width=0.8\textwidth]{figures/age-sex_data_analysis/Male.png}
	\caption{CI des taux d'incidence des MRC par âge pour les mâles.}
	\label{fig:CI-age-sex-males}
\end{figure}

\begin{figure}[H]
	\centering
	\includegraphics[width=0.8\textwidth]{figures/age-sex_data_analysis/Female.png}
	\caption{CI des taux d'incidence des MRC par âge pour les femelles.}
	\label{fig:CI-age-sex-females}
\end{figure}

\subsection{Tranjectoire de vie dans le plan de Poincaré}

Les trajectoires dans le demi-plan de Poincaré des paramètres ($\mu$, $\sigma$) par âge et sexe sont présentées dans les Figures  \ref{fig:age-sex_Poincare_male} et  \ref{fig:age-sex_Poincare_female}.

\begin{figure}[H]
\centering
\includegraphics[width=0.8\textwidth]{figures/age-sex_Poincare/US-Male.png}
\caption{Trajectoires hyperboliques des paramètres ($\mu$, $\sigma$) par âge pour les mâles.}
\label{fig:age-sex_Poincare_male}
\end{figure}

\begin{figure}[H]
	\centering
	\includegraphics[width=0.8\textwidth]{figures/age-sex_Poincare/US-Female.png}
	\caption{Trajectoires hyperboliques des paramètres ($\mu$, $\sigma$) par âge pour les femelles.}
	\label{fig:age-sex_Poincare_female}
\end{figure}

\subsection{Distance de Fisher-Rao}

La somme cumulée des distances de Fisher-Rao le long des trajectoires par sexe pour chaque pays est présentée dans la Figure \ref{fig:age-sex_FR_cumulative}.

\begin{figure}[H]
\centering
\includegraphics[width=0.8\textwidth]{figures/age-sex_distance_cumulative/Fisher-Rao.png}
\caption{Somme cumulée des distances de Fisher-Rao le long des trajectoires d'âge, stratifiées par sexe.}
\label{fig:age-sex_FR_cumulative}
\end{figure}

\subsection{Divergence de Kullback-Leibler}

La somme cumulée des divergences de Kullback-Leibler le long des trajectoires par sexe pour chaque pays est présentée dans la Figure \ref{fig:age-sex_KL_cumulative}.

\begin{figure}[H]
	\centering
	\includegraphics[width=0.8\textwidth]{figures/age-sex_distance_cumulative/Kullback-Leibler.png}
	\caption{Somme cumulée des divergences de Kullback-Leibler le long des trajectoires d'âge, stratifiées par sexe.}
	\label{fig:age-sex_KL_cumulative}
\end{figure}

\section{L'effet du sexe stratifié par l'âge}

\subsection{Dynamique hyperbolique}

La trajectoire dans le demi-plan de Poincaré des paramètres ($\mu$, $\sigma$) de $g$ par âge est présentée à la Figure \ref{fig:poincare-sexe-age}.

%\begin{figure}[H]
%\centering
%\includegraphics[width=0.8\textwidth]{figures/poincare_sexe_age/poincare_sexe_age.png}
%\caption{Trajectoire hyperbolique des paramètres de $g$ par âge.}
%\label{fig:poincare-sexe-age}
%\end{figure}
image ici

\subsection{Distance de Fisher-Rao}

La longueur cumulée des trajectoires de Fisher-Rao par pays est présentée dans le tableau suivant.

%\input{tables/fr_cumule_age.tex}
table ici

\subsection{Divergence de Kullback-Leibler}

La somme cumulée des divergences de Kullback-Leibler par pays est présentée dans le tableau suivant.

%\input{tables/kl_cumule_age.tex}
table ici



\section{L'effet de l'arsenic PM$_{2.5}$}

\subsection{Distribution géographique}

Les paramètres $\mu_s$ sont visualisés sur la carte de la Figure \ref{fig:arsenic-map}, qui illustre les concentrations moyennes par État. Les paramètres $(\mu_s, \sigma_s)$ sont représentés dans la Figure \ref{fig:arsenic_Poincare}.

\begin{figure}[H]
	\centering
	\includegraphics[width=1\textwidth]{figures/arsenic_map/arsenic_map.png}
	\caption{Concentrations moyennes d'arsenic PM$_{2.5}$ par État aux États-Unis (1990-2019).}
	\label{fig:arsenic-map}
\end{figure}

\begin{figure}[H]
	\centering
	\includegraphics[width=1\textwidth]{figures/arsenic_Poincare/arsenic_Poincare.png}
	\caption{Paramètres $(\mu_s, \sigma_s)$ par État aux États-Unis (1990-2019).}
	\label{fig:arsenic_Poincare}
\end{figure}

Les paramètres $\mu_{s,a,g}$ sont visualisés sur la carte de la Figure \ref{fig:GBD-map}, qui illustre les concentrations moyennes par État. Les paramètres $(\mu_{s,a,g}$, $\sigma_{s,a,g})$ sont représentés dans la Figure \ref{fig:GBD_Poincare}.

\begin{figure}[H]
	\centering
	\includegraphics[width=1\textwidth]{figures/GBD_map/GBD_map.png}
	\caption{Log-cotes de la prévalence des MRC par État aux États-Unis (1990-2019).}
	\label{fig:GBD-map}
\end{figure}

\begin{figure}[H]
	\centering
	\includegraphics[width=1\textwidth]{figures/GBD_Poincare/GBD_Poincare.png}
	\caption{Paramètres $(\mu_{s,a,g}, \sigma_{s,a,g})$ par État aux États-Unis (1990-2019).}
	\label{fig:GBD_Poincare}
\end{figure}

\subsection{Intervales de confiance}

Les intervales de confiance des $\hat{\beta}_{a,g}$ sont présentés dans les Figures \ref{fig:CI-arsenic-males} et \ref{fig:CI-arsenic-females}.

\begin{figure}[H]
	\centering
	\includegraphics[width=0.8\textwidth]{figures/arsenic_data_analysis/Male.png}
	\caption{CI des pentes par âge pour les mâles..}
	\label{fig:CI-arsenic-males}
\end{figure}

\begin{figure}[H]
	\centering
	\includegraphics[width=0.8\textwidth]{figures/arsenic_data_analysis/Female.png}
	\caption{CI des pentes par âge pour les femelles.}
	\label{fig:CI-arsenic-females}
\end{figure}

\subsection{Tranjectoire de vie dans le plan de Poincaré}

Les trajectoires dans le demi-plan de Poincaré des $\hat{\beta}_{a,g}$ + $i \cdot \mathrm{MAD}_{a,g}$ par âge et sexe, les sommes cumulées des distances de Fisher-Rao et divergences de Kullback-Leibler par sexe, ainsi que les trajectoires agrégées par clusters de régions ordonnées par $\mu_s$ sont présentées dans les Figures \ref{fig:trajectoires-arsenic-males} et \ref{fig:trajectoires-arsenic-females}.

\begin{figure}[H]
\centering
\includegraphics[width=0.8\textwidth]{figures/arsenic_Poincare_slopes/Male.png}
\caption{Trajectoires des effets de l'arsenic sur l'incidence des MRC par âge pour les mâles.}
\label{fig:trajectoires-arsenic-males}
\end{figure}

\begin{figure}[H]
	\centering
	\includegraphics[width=0.8\textwidth]{figures/arsenic_Poincare_slopes/Female.png}
	\caption{Trajectoires des effets de l'arsenic sur l'incidence des MRC par âge pour les femelles.}
	\label{fig:trajectoires-arsenic-females}
\end{figure}

\subsection{Distance de Fisher Rao cumulée}

Les sommes cumulées des distances de Fisher-Rao pour les trajectoires des effets de l'arsenic sont présentées dans le Figures \ref{fig:arsenic_distance_cumulative:Fisher-Rao}.

\begin{figure}[H]
	\centering
	\includegraphics[width=0.8\textwidth]{figures/arsenic_distance_cumulative/Fisher-Rao.png}
	\caption{Distance de Fisher-Rao cumulée.}
	\label{fig:arsenic_distance_cumulative:Fisher-Rao}
\end{figure}

\subsection{Distance de Fisher Rao cumulée}

Les sommes cumulées des divergences de Kullback-Leibler pour les trajectoires des effets de l'arsenic sont présentées dans le Figures \ref{fig:arsenic_distance_cumulative:Kullback-Leibler}.

\begin{figure}[H]
	\centering
	\includegraphics[width=0.8\textwidth]{figures/arsenic_distance_cumulative/Kullback-Leibler.png}
	\caption{Divergence de Kullback-Leibler cumulée.}
	\label{fig:arsenic_distance_cumulative:Kullback-Leibler}
\end{figure}